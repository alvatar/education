\documentclass[a4paper,oneside]{article}
% Graphics
\usepackage{graphicx}
\usepackage{wrapfig}
% Font selection
\usepackage[utf8]{inputenc}
\usepackage{PTSansNarrow} 
\renewcommand*\familydefault{\sfdefault} %% Only if the base font of the document is to be sans serif
\usepackage[T1]{fontenc}
% Code highlighting
\usepackage{minted}
% Links
\usepackage[hidelinks]{hyperref}

% Background color definition
\definecolor{bg}{rgb}{0.95,0.95,0.95}

% Code line numbers configuration
\renewcommand{\theFancyVerbLine}{\sffamily
  \textcolor[rgb]{0.5,0.5,1.0}{\scriptsize
     \oldstylenums{\arabic{FancyVerbLine}}}}


\makeatletter
\newcommand{\minted@write@detok}[1]{%
  \immediate\write\FV@OutFile{\detokenize{#1}}}%

\newcommand{\minted@FVB@VerbatimOut}[1]{%
  \@bsphack
  \begingroup
    \FV@UseKeyValues
    \FV@DefineWhiteSpace
    \def\FV@Space{\space}%
    \FV@DefineTabOut
    %\def\FV@ProcessLine{\immediate\write\FV@OutFile}% %Old, non-Unicode version
    \let\FV@ProcessLine\minted@write@detok %Patch for Unicode
    \immediate\openout\FV@OutFile #1\relax
    \let\FV@FontScanPrep\relax
%% DG/SR modification begin - May. 18, 1998 (to avoid problems with ligatures)
    \let\@noligs\relax
%% DG/SR modification end
    \FV@Scan}
    \let\FVB@VerbatimOut\minted@FVB@VerbatimOut

\renewcommand\minted@savecode[1]{
  \immediate\openout\minted@code\jobname.pyg
  \immediate\write\minted@code{\expandafter\detokenize\expandafter{#1}}%
  \immediate\closeout\minted@code}
\makeatother


\title{- VIII -\linebreak Movimiento (I)}
\author{U-tad\\ Diseño Visual de Contenidos Digitales\\ Prof. Álvaro Castro-Castilla}
\date{}

\begin{document}
\maketitle


\section{Conceptos que emplearemos}
\begin{enumerate}
  \item Movimiento

    \begin{minted}[linenos=true,bgcolor=bg]{java}
// Empezamos en posición X = 0
int positionX = 0;
// La velocidad de avance será de 1 pixel por fotograma
int speedX = 1;

void draw() {
  /* La velocidad es el cambio de posición en el tiempo,
     por lo tanto tendremos que aplicar ese cambio en cada
     fotograma, sumando la velocidad a la posición, para
     obtener la nueva posición en cada fotograma.
     Si la velocidad es 1 pixel/fotograma, en cada
     fotograma la posición será 1 pixel más.  */
  positionX += speedX;
}
    \end{minted}

  \item Dirección del movimiento

    \begin{minted}[linenos=true,bgcolor=bg]{java}
int positionX = 0;
int positionY = 0;
int speedX = 1;
int speedY = 1;

void draw() {
  // Velocidad en el eje X
  positionX += speedX;
  // Velocidad en el eje Y
  positionY += speedY;
}
    \end{minted}

\newpage
  \item Movimiento con precisión

    \begin{minted}[linenos=true,bgcolor=bg]{java}
float positionX = 0.0;
float positionY = 0.0;
float speedX = 0.5;
float speedY = 0.2;

void draw() {
  // Velocidad en el eje X (0.5 pixeles/fotograma)
  positionX += speedX;
  // Velocidad en el eje Y (0.2 pixeles/fotograma)
  positionY += speedY;
}
    \end{minted}

  \item Interpolación del movimiento

    \begin{minted}[linenos=true,bgcolor=bg]{java}
// Coordenada X inicial
int startX = 0;
// Coordenada X final
int stopX = 200;

// Coordenada X actual
float currentX = startX;

// Incremento por cada fotograma
float step = 0.005;
// Porcentage de la animación total (de 0.0 a 1.0)
float percentage = 0.0;

void setup() {
  size(500, 500);
  smooth();
}

void draw() {
  background(0);

  if (percentage < 1.0) {
    x = startX + ((stopX-startX) * percentage);
    y = startY + ((stopY-startX) * percentage);
    percentage += step;
  }

  ellipse(currentX, 200, 20, 20);
}
    \end{minted}

\newpage
  \item Ciclos y rebotes en el movimiento

    \begin{minted}[linenos=true,bgcolor=bg]{java}
float x = 0;
float speed = 0.5;
/* Si bounce es false, el círculo comenzará otra vez desde
   la izquierda */
boolean bounce = true;

void setup() {
  size(500, 500);
}

void draw() {
  background(0);
  x += speed;

  // Cuando la posición "x" sea mayor que el ancho
  // recomenzar con x = 0
  if (x > width) {
    if (bounce) {
      // Rebote...
      speed = -speed;
    } else {
      // ...o recomenzar desde la izquierda
      x = 0;
    }
  }

  ellipse(x, 100, 40, 40);
}
    \end{minted}

  \item Temporizadores

    \begin{minted}[linenos=true,bgcolor=bg]{java}
int timer = 2000;

void setup() {
  size(500, 500);
}

void draw() {
  // Calcular tiempo actual (desde comienzo de programa)
  int currentTime = millis();

  background(0);

  /* Si el tiempo actual es mayor que el temporizador,
     realizar la acción deseada */
  if (currentTime > timer) {
    ellipse(x, 60, 90, 90);
  }
}
    \end{minted}

      
      
\end{enumerate}


\section{Elementos de código que emplearemos}
\begin{enumerate}
  \item Ver y definir el rendimiento/velocidad de renderizado

    \begin{minted}[linenos=true,bgcolor=bg]{java}
void setup() {
  // Definir la velocidad deseada a 30 fotogr./seg.
  frameRate(30);
}

void draw() {
  // Mostrar la velocidad real
  println(frameRate);
}
    \end{minted}
\end{enumerate}

\end{document}
