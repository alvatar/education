\documentclass[a4paper,oneside]{article}
% Graphics
\usepackage{graphicx}
\usepackage{wrapfig}
% Font selection
\usepackage[utf8]{inputenc}
\usepackage{PTSansNarrow} 
\renewcommand*\familydefault{\sfdefault} %% Only if the base font of the document is to be sans serif
\usepackage[T1]{fontenc}
% Code highlighting
\usepackage{minted}
% Links
\usepackage[hidelinks]{hyperref}

% Background color definition
\definecolor{bg}{rgb}{0.95,0.95,0.95}

% Code line numbers configuration
\renewcommand{\theFancyVerbLine}{\sffamily
  \textcolor[rgb]{0.5,0.5,1.0}{\scriptsize
     \oldstylenums{\arabic{FancyVerbLine}}}}


\makeatletter
\newcommand{\minted@write@detok}[1]{%
  \immediate\write\FV@OutFile{\detokenize{#1}}}%

\newcommand{\minted@FVB@VerbatimOut}[1]{%
  \@bsphack
  \begingroup
    \FV@UseKeyValues
    \FV@DefineWhiteSpace
    \def\FV@Space{\space}%
    \FV@DefineTabOut
    %\def\FV@ProcessLine{\immediate\write\FV@OutFile}% %Old, non-Unicode version
    \let\FV@ProcessLine\minted@write@detok %Patch for Unicode
    \immediate\openout\FV@OutFile #1\relax
    \let\FV@FontScanPrep\relax
%% DG/SR modification begin - May. 18, 1998 (to avoid problems with ligatures)
    \let\@noligs\relax
%% DG/SR modification end
    \FV@Scan}
    \let\FVB@VerbatimOut\minted@FVB@VerbatimOut

\renewcommand\minted@savecode[1]{
  \immediate\openout\minted@code\jobname.pyg
  \immediate\write\minted@code{\expandafter\detokenize\expandafter{#1}}%
  \immediate\closeout\minted@code}
\makeatother


\title{- XII -\linebreak Clases y Objetos}
\author{U-tad\\ Diseño Visual de Contenidos Digitales\\ Prof. Álvaro Castro-Castilla}
\date{}

\begin{document}
\maketitle


\section{Tutoriales online relacionados}
\begin{enumerate}
  \item \href{http://processing.org/learning/objects/}{Tutorial Programación Orientada a Objetos con Processing}
\end{enumerate}


\section{Conceptos que emplearemos}
\begin{enumerate}

  \item Instanciar \textbf{(crear)} un objeto

    \begin{minted}[linenos=true,bgcolor=bg]{java}
// Crear una instancia de la clase PImage
PImage myImage;

// Crear una instancia de la clase PFont
PFont myFont;

// Aunque no son objetos, también son tipos (de datos),
// con lo que se instancian de la misma manera
int myInt;

float myFloat;

/* String también es una clase, para trabajar con
   cadenas de caracteres */
String myString;

    \end{minted}

\newpage
  \item Programación empleando funciones y variables

    \begin{minted}[linenos=true,bgcolor=bg]{java}
// Posición del Alien
int posX;
int posY;

void setup() {
  size(500, 500);
  // Inicialización de las variables de posición
  posX = width/2;
  posY = height/2;
}

// Función para dibujar el Alien
void displayAlien() {
  ellipse(posX-10, posY-10, 20, 20);
  ellipse(posX+10, posY-10, 20, 20);
  ellipse(posX, posY-10, 10, 10);
  ellipse(posX, posY, 20, 20);
}

// Función para mover el Alien
void moveAlien(float speedX) {
  posX += speedX;
  if (posX > width) {
    posX = 0;
  }
}

void draw() {
  background(255);
  // Llamar a las funciones de dibujar y mover el Alien
  displayAlien();
  moveAlien(1.5);
}
    \end{minted}

\newpage
  \item Programación del mismo código con la clase Alien

    \begin{minted}[linenos=true,bgcolor=bg]{java}
// La clase Alien
class Alien {
  // Atributos de la clase Alien
  int posX;
  int posY;

  // Constructor de la clase
  Alien() {
    // Inicialización de las variables
    posX = width/2;
    posY = height/2;
  }

  // Método para mostrar el Alien
  void display() {
    ellipse(posX-10, posY-10, 20, 20);
    ellipse(posX+10, posY-10, 20, 20);
    ellipse(posX, posY-10, 10, 10);
    ellipse(posX, posY, 20, 20);
  }
  
  // Método para mover el Alien
  void move(float speedX) {
    posX += speedX;
    if (posX > width) {
      posX = 0;
    }
  }
}

// Creación de una instancia de Alien
Alien myAlien;

void setup() {
  size(500, 500);
  // Inicialización de la instancia myAlien
  myAlien = new Alien();
}

void draw() {
  background(255);
  myAlien.display();
  myAlien.move(1.5);
}
    \end{minted}
\end{enumerate}

\newpage

\section{Elementos de código que emplearemos}
\begin{enumerate}

  \item Definición de una clase

    \begin{minted}[linenos=true,bgcolor=bg]{java}
class Car {
  // Todos los elementos de la clase vienen aquí
}
    \end{minted}

  \item Constructor

    \begin{minted}[linenos=true,bgcolor=bg]{java}
class Car {
  // Esta función es el constructor de la clase
  Car() {
  }
}
    \end{minted}

  \item Atributos de la clase (variables)

    \begin{minted}[linenos=true,bgcolor=bg]{java}
class Car {
  // Datos internos de la clase (atributos)
  int x, y;
  String name;

  Car() {
  }
}
    \end{minted}

  \item Métodos de la clase (funciones)

    \begin{minted}[linenos=true,bgcolor=bg]{java}
class Car {
  int x, y;
  String name;

  Car() {
  }

  /* Función move: mueve el coche
     No recibe ningún parámetro
     No devuelve ningún parámetro */
  void move() {
    x++; y++;
  }

  /* Función changeName: cambia el nombre del coche
     Recibe un parámetro: el nuevo nombre
     No devuelve ningún parámetro */
  void changeName(String newName) {
    name = newName;
  }
}
    \end{minted}

\newpage

  \item Creación de una instancia de la clase Car

    \begin{minted}[linenos=true,bgcolor=bg]{java}
// 1. Declarar una variable
Car myCar;

void setup() {
  // 2. Inicializar la variable
  myCar = new Car();
}

void draw() {
  // 3. Llamar a métodos de la instancia
  myCar.move();
}
    \end{minted}

  \item Constructores con parámetros

    \begin{minted}[linenos=true,bgcolor=bg]{java}
class Car{
  int x;

  // Los parámetros se definen como en una función
  Car(int newX) {
    x = newX;
  }
}

// Inicialización de la instancia de Car con el parámetro
Car myCar = new Car(100);
    \end{minted}

\end{enumerate}

\end{document}
