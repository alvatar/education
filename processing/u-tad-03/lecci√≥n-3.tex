\documentclass[a4paper,oneside]{article}
% Graphics
\usepackage{graphicx}
\usepackage{wrapfig}
% Font selection
\usepackage[utf8]{inputenc}
\usepackage{PTSansNarrow} 
\renewcommand*\familydefault{\sfdefault} %% Only if the base font of the document is to be sans serif
\usepackage[T1]{fontenc}
% Code highlighting
\usepackage{minted}
% Links
\usepackage[hidelinks]{hyperref}

% Background color definition
\definecolor{bg}{rgb}{0.95,0.95,0.95}

% Code line numbers configuration
\renewcommand{\theFancyVerbLine}{\sffamily
  \textcolor[rgb]{0.5,0.5,1.0}{\scriptsize
     \oldstylenums{\arabic{FancyVerbLine}}}}


\makeatletter
\newcommand{\minted@write@detok}[1]{%
  \immediate\write\FV@OutFile{\detokenize{#1}}}%

\newcommand{\minted@FVB@VerbatimOut}[1]{%
  \@bsphack
  \begingroup
    \FV@UseKeyValues
    \FV@DefineWhiteSpace
    \def\FV@Space{\space}%
    \FV@DefineTabOut
    %\def\FV@ProcessLine{\immediate\write\FV@OutFile}% %Old, non-Unicode version
    \let\FV@ProcessLine\minted@write@detok %Patch for Unicode
    \immediate\openout\FV@OutFile #1\relax
    \let\FV@FontScanPrep\relax
%% DG/SR modification begin - May. 18, 1998 (to avoid problems with ligatures)
    \let\@noligs\relax
%% DG/SR modification end
    \FV@Scan}
    \let\FVB@VerbatimOut\minted@FVB@VerbatimOut

\renewcommand\minted@savecode[1]{
  \immediate\openout\minted@code\jobname.pyg
  \immediate\write\minted@code{\expandafter\detokenize\expandafter{#1}}%
  \immediate\closeout\minted@code}
\makeatother


\title{- III -\linebreak Variables}
\author{U-tad\\ Diseño Visual de Contenidos Digitales\\ Prof. Álvaro Castro-Castilla}
\date{}

\begin{document}
\maketitle


\section{Conceptos que emplearemos}
\begin{enumerate}
  \item Definición de una variable

    \begin{minted}[linenos=true,bgcolor=bg]{java}
/* Declarar una variable es definir su "tipo" y su
 * "nombre"
 *
 * Tipo de la variable: int (número entero)
 * Nombre de la variable: mi-variable */
int mi-variable;

/* Asignar un valor a una variable es ponerle
 * contenido.
 *
 * El contenido tiene que ser del tipo que hemos
 * declarado anteriormente: si la variable es de
 * tipo "int", el contenido tendrá que ser un
 * número entero */
mi-variable = 50;

/* Podemos hacer las dos cosas al mismo tiempo:
 * definir la variable y asignarle valor */
int mi-variable = 50;
    \end{minted}

  \item Realizar operaciones y asignar a las variables

    \begin{minted}[linenos=true,bgcolor=bg]{java}
// "x" vale 20
int x = 20;

// "y" valdrá 30
int y = x + 10;

// "z" valdrá 30 + 20
int z = x + y;
    \end{minted}

  \newpage
  \item Agrupación de las operaciones

    \begin{minted}[linenos=true,bgcolor=bg]{java}
/* Lo que está entre paréntesis se realiza primero:
 * En este caso el resultado será igual a 10 * 2 */
int x = (4 + 6) * 2;

/* En estos dos casos será igual a 4 + 12. El segundo
 * de ellos por la precedencia de la multiplicación
 * sobre la suma (como en matemáticas) */
int x = 4 + (6 * 2);
int x = 4 + 6 * 2;
    \end{minted}

  \item Orden de las definiciones

    \begin{minted}[linenos=true,bgcolor=bg]{java}
/* Error: usamos "y" antes de definirla,
 * la variable todavía no existe */
int x = y + 20;
int y = 10;
    \end{minted}

  \item Ámbito de las definiciones (scope)

    \begin{minted}[linenos=true,bgcolor=bg]{java}
// Variable global (definida fuera de setup/draw)
int globalVar = 10;

void setup() {
  // Variable local de setup
  int localVarSetup = 20;
}

void draw() {
  // Variable local de draw
  int localVarDraw = 30;

  /* Es posible usar la variable global y la local
   * a tu ámbito */
  int y = globalVar + localVarDraw;

  // ...pero no la variable local de otro ámbito
  int z = localVarSetup; // Error!!
}
    \end{minted}

\end{enumerate}


\newpage
\section{Elementos de código que emplearemos}
\begin{enumerate}
  \item Tipos de datos "números enteros" (\textbf{INT}eger)

    \mint[bgcolor=bg]{java} |int|
  \item Tipos de datos "coma flotante", es decir, con decimales (\textbf{FLOAT}ing point)

    \mint[bgcolor=bg]{java} |float|
  \item Suma

    \mint[bgcolor=bg]{java} |1 + 199|
  \item Resta

    \mint[bgcolor=bg]{java} |27 - 7|
  \item Multiplicación

    \mint[bgcolor=bg]{java} |33 * 29|
  \item División

    \mint[bgcolor=bg]{java} |121 / 11|
  \item Abreviaturas

    \begin{minted}[linenos=true,bgcolor=bg]{java}
/* Sumar el valor 2 y asignar el resultado
 * (o cualquier otro) a la variable "x"
 * Equivale a x = x + 2; */
x += 2;
/* Restar el valor 2 y asignar el resultado
 * (o cualquier otro) a la variable "x"
 * Equivale a x = x - 2; */
x -= 2;

/* Sumar 1 y asignar el resultado
 * x = x + 1;
 * Muy útil para incrementar un valor
 * paso a paso */
x++;

/* Restar 1 y asignar el resultado
 * x = x - 1;
 * Muy útil para decrementar un valor
 * paso a paso */
x--;
    \end{minted}
\end{enumerate}

\end{document}
