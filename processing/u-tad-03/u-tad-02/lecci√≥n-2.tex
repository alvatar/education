\documentclass[a4paper,oneside]{article}
% Graphics
\usepackage{graphicx}
\usepackage{wrapfig}
% Font selection
\usepackage[utf8]{inputenc}
\usepackage{PTSansNarrow} 
\renewcommand*\familydefault{\sfdefault} %% Only if the base font of the document is to be sans serif
\usepackage[T1]{fontenc}
% Code highlighting
\usepackage{minted}
% Links
\usepackage[hidelinks]{hyperref}

% Background color definition
\definecolor{bg}{rgb}{0.95,0.95,0.95}

% Code line numbers configuration
\renewcommand{\theFancyVerbLine}{\sffamily
  \textcolor[rgb]{0.5,0.5,1.0}{\scriptsize
     \oldstylenums{\arabic{FancyVerbLine}}}}


\makeatletter
\newcommand{\minted@write@detok}[1]{%
  \immediate\write\FV@OutFile{\detokenize{#1}}}%

\newcommand{\minted@FVB@VerbatimOut}[1]{%
  \@bsphack
  \begingroup
    \FV@UseKeyValues
    \FV@DefineWhiteSpace
    \def\FV@Space{\space}%
    \FV@DefineTabOut
    %\def\FV@ProcessLine{\immediate\write\FV@OutFile}% %Old, non-Unicode version
    \let\FV@ProcessLine\minted@write@detok %Patch for Unicode
    \immediate\openout\FV@OutFile #1\relax
    \let\FV@FontScanPrep\relax
%% DG/SR modification begin - May. 18, 1998 (to avoid problems with ligatures)
    \let\@noligs\relax
%% DG/SR modification end
    \FV@Scan}
    \let\FVB@VerbatimOut\minted@FVB@VerbatimOut

\renewcommand\minted@savecode[1]{
  \immediate\openout\minted@code\jobname.pyg
  \immediate\write\minted@code{\expandafter\detokenize\expandafter{#1}}%
  \immediate\closeout\minted@code}
\makeatother


\title{- II -\linebreak Propiedades de las herramientas de dibujo}
\author{U-tad\\ Diseño Visual de Contenidos Digitales\\ Prof. Álvaro Castro-Castilla}
\date{}

\begin{document}
\maketitle


\section{Conceptos que emplearemos}
\begin{enumerate}
  \item Estado de las herramientas de dibujo

    \begin{minted}[linenos=true,bgcolor=bg]{java}
void draw() {
  // Establecer el grosor de pintado a 10
  strokeWeight(10);
  // El círculo se pintará con grosor 10
  ellipse(5, 50, 60, 60);

  // Aquí el "estado" del grosor de pintado sigue siendo 10

  // Aquí modificamos el estado del grosor de pintado a 20
  strokeWeight(20);
  // El círculo se pintará con grosor 20
  ellipse(105, 50, 60, 60);

  /* Aquí el "estado" del grosor de pintado es 20...
     y cuando volvamos a pintar desde el principio,
     permanecerá así. */
}
    \end{minted}

  \item Formato de los colores

    \textbf{[0 al 255]} se interpreta como: Gris

    \textbf{[0 al 255], [0 al 255]} se interpreta como: Gris + Alpha (transparencia)
    
    \textbf{[0 al 255], [0 al 255], [0 al 255]} se interpreta como: RGB (Red, Green, Blue)

    \textbf{[0 al 255], [0 al 255], [0 al 255], [0 al 255]} se interpreta como: RGBA (Red, Green, Blue, Alpha)

    Mediante esta línea de código podemos hacer que la codificación no sea RGB, sino HSB (Hue, Saturation, Brightness)

    \mint[bgcolor=bg]{java} |colorMode(HSB);|

    Más opciones de \textit{colorMode()} se pueden ver en la referencia de Processing
\end{enumerate}


\section{Elementos de código que emplearemos}
\begin{enumerate}
  \item Dibujar líneas con suavizado

    \mint[bgcolor=bg]{java} |smooth();|
  \item Desactivar el suavizado

    \mint[bgcolor=bg]{java} |noSmooth();|
  \item Establecer el grosor de pintado

    \mint[bgcolor=bg]{java} |strokeWeight(...);|
  \item Atributos de línea

    \begin{minted}[bgcolor=bg]{java}
strokeJoin(...);
strokeCap(...);
    \end{minted}
  \item Pintar el fondo de un color

    \mint[bgcolor=bg]{java} |background(...);|
  \item Elegir un color de pluma

    \mint[bgcolor=bg]{java} |stroke(...);|
  \item No usar ningún color de pluma

    \mint[bgcolor=bg]{java} |noStroke(...);|
  \item Elegir un color de relleno

    \mint[bgcolor=bg]{java} |fill(...);|
  \item No usar ningún color de relleno

    \mint[bgcolor=bg]{java} |noFill(...);|
  \item Pintar el fondo de un color

    \mint[bgcolor=bg]{java} |background(...);|
\end{enumerate}

\end{document}
