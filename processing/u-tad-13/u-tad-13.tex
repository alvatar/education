\documentclass[a4paper,oneside]{article}
% Graphics
\usepackage{graphicx}
\usepackage{wrapfig}
% Font selection
\usepackage[utf8]{inputenc}
\usepackage{PTSansNarrow} 
\renewcommand*\familydefault{\sfdefault} %% Only if the base font of the document is to be sans serif
\usepackage[T1]{fontenc}
% Code highlighting
\usepackage{minted}
% Links
\usepackage[hidelinks]{hyperref}

% Background color definition
\definecolor{bg}{rgb}{0.95,0.95,0.95}

% Code line numbers configuration
\renewcommand{\theFancyVerbLine}{\sffamily
  \textcolor[rgb]{0.5,0.5,1.0}{\scriptsize
     \oldstylenums{\arabic{FancyVerbLine}}}}


\makeatletter
\newcommand{\minted@write@detok}[1]{%
  \immediate\write\FV@OutFile{\detokenize{#1}}}%

\newcommand{\minted@FVB@VerbatimOut}[1]{%
  \@bsphack
  \begingroup
    \FV@UseKeyValues
    \FV@DefineWhiteSpace
    \def\FV@Space{\space}%
    \FV@DefineTabOut
    %\def\FV@ProcessLine{\immediate\write\FV@OutFile}% %Old, non-Unicode version
    \let\FV@ProcessLine\minted@write@detok %Patch for Unicode
    \immediate\openout\FV@OutFile #1\relax
    \let\FV@FontScanPrep\relax
%% DG/SR modification begin - May. 18, 1998 (to avoid problems with ligatures)
    \let\@noligs\relax
%% DG/SR modification end
    \FV@Scan}
    \let\FVB@VerbatimOut\minted@FVB@VerbatimOut

\renewcommand\minted@savecode[1]{
  \immediate\openout\minted@code\jobname.pyg
  \immediate\write\minted@code{\expandafter\detokenize\expandafter{#1}}%
  \immediate\closeout\minted@code}
\makeatother


\title{- XIII -\linebreak Arrays}
\author{U-tad\\ Diseño Visual de Contenidos Digitales\\ Prof. Álvaro Castro-Castilla}
\date{}

\begin{document}
\maketitle


\section{Tutoriales online relacionados}
\begin{enumerate}
  \item \href{http://processing.org/learning/2darray/}{Arrays bidimensionales} 
\end{enumerate}


\section{Conceptos que emplearemos}
\begin{enumerate}
  \item Programar con múltiples variables

    \begin{minted}[linenos=true,bgcolor=bg]{java}
float rect1x = 0;
float rect2x = 10;
float rect3x = 45;
float rect4x = -18;
float rect5x = 30;

void setup() {
  size(500, 500);
}

void draw() {
  background(255);
  // Dibujar y modificar cada rectángulo por su lado
  rect(rect1x, 160, 20, 20);
  rect(rect2x, 220, 20, 20);
  rect(rect3x, 260, 20, 20);
  rect(rect4x, 300, 20, 20);
  rect(rect5x, 340, 20, 20);
  rect1x += 0.5;
  rect2x += 0.5;
  rect3x += 0.5;
  rect4x += 0.5;
  rect5x += 0.5;
}
    \end{minted}

\newpage

  \item Programar con arrays

    \begin{minted}[linenos=true,bgcolor=bg]{java}
// Declarar y definir los arrays
float[] rectXs = {0, 10, 45, -18, 30};
float[] rectYs = {160, 220, 260, 300, 340};

void setup() {
  size(500, 500);
}

void draw() {
  background(255);
  // Dibujar y modificar cada rectángulo
  for (int i = 0; i < rectXs.length; i++) {
    rect(rectXs[i], rectYs[i], 20, 20);
    rectXs[i] += 0.5;
  }
}
    \end{minted}

\newpage

  \item Programar con arrays de objetos

    \begin{minted}[linenos=true,bgcolor=bg]{java}
class Alien {
  // Atributos del alien
  float x, y;
  
  // Constructor del alien
  Alien(float startX, float startY) {
    x = startX;
    y = startY;
  }
  
  // Dibujar el alien (¡ahora sólo es un rect!)
  void display(float sizeX, float sizeY) {
    rect(x, y, sizeX, sizeY);
  }
  
  // Mover el alien
  void move(float deltaX, float deltaY) {
    x += deltaX;
    y += deltaY;
  }
}

// Declarar el array de aliens sin inicializar
Alien[] aliens;

void setup() {
  size(500, 500);
  // Inicializar el array de aliens
  aliens = new Alien[5];
  // Inicializar cada alien
  aliens[0] = new Alien(0, 160);
  aliens[1] = new Alien(10, 220);
  aliens[2] = new Alien(45, 260);
  aliens[3] = new Alien(-18, 300);
  aliens[4] = new Alien(30, 340);
}

void draw() {
  background(255);
  // Dibujar y modificar cada rectángulo
  for (int i = 0; i < aliens.length; i++) {
    aliens[i].display(20, 20);
    aliens[i].move(0.5, 0.0);
  }
}
    \end{minted}

\end{enumerate}


\section{Elementos de código que emplearemos}
\begin{enumerate}

  \item Declarar un array

    \begin{minted}[linenos=true,bgcolor=bg]{java}
// Declarar un array de floats
float[] myFloats;

// Declarar un array de Strings
String[] myStrings;

// Declarar un array de Aliens
Alien[] myAliens;
    \end{minted}

  \item Definir un array de tipos simples

    \begin{minted}[linenos=true,bgcolor=bg]{java}
// Sin inicializar los valores del array
myFloats = new float[3];
// Inicializando los valores del array
myFloats = {1,2,3};
myFloats = new float[]{1,2,3};
    \end{minted}

  \item Definir un array de objetos (1)

    \begin{minted}[linenos=true,bgcolor=bg]{java}
// Sin inicializar los objetos del array
myStrings = new String[3];
// Inicializando los objetos del array
myStrings = {"a","b","c"};
myStrings = new String[]{"a","b","c"};
    \end{minted}

  \item Definir un array de objetos (2)

    \begin{minted}[linenos=true,bgcolor=bg]{java}
// Sin inicializar los objetos del array
myAliens = new Alien[3];
// Inicializando los objetos del array
aliens = new Alien[]{
  new Alien(0, 160),
  new Alien(10, 220),
  new Alien(45, 260)
};
// Inicializando los objetos uno a uno
myAliens = new Alien[3];
myAliens[0] = new Alien(0,160);
myAliens[1] = new Alien(10,220);
myAliens[2] = new Alien(45,260);
    \end{minted}

\end{enumerate}

\end{document}
