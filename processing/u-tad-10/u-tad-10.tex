\documentclass[a4paper,oneside]{article}
% Graphics
\usepackage{graphicx}
\usepackage{wrapfig}
% Font selection
\usepackage[utf8]{inputenc}
\usepackage{PTSansNarrow} 
\renewcommand*\familydefault{\sfdefault} %% Only if the base font of the document is to be sans serif
\usepackage[T1]{fontenc}
% Code highlighting
\usepackage{minted}
% Links
\usepackage[hidelinks]{hyperref}

% Background color definition
\definecolor{bg}{rgb}{0.95,0.95,0.95}

% Code line numbers configuration
\renewcommand{\theFancyVerbLine}{\sffamily
  \textcolor[rgb]{0.5,0.5,1.0}{\scriptsize
     \oldstylenums{\arabic{FancyVerbLine}}}}


\makeatletter
\newcommand{\minted@write@detok}[1]{%
  \immediate\write\FV@OutFile{\detokenize{#1}}}%

\newcommand{\minted@FVB@VerbatimOut}[1]{%
  \@bsphack
  \begingroup
    \FV@UseKeyValues
    \FV@DefineWhiteSpace
    \def\FV@Space{\space}%
    \FV@DefineTabOut
    %\def\FV@ProcessLine{\immediate\write\FV@OutFile}% %Old, non-Unicode version
    \let\FV@ProcessLine\minted@write@detok %Patch for Unicode
    \immediate\openout\FV@OutFile #1\relax
    \let\FV@FontScanPrep\relax
%% DG/SR modification begin - May. 18, 1998 (to avoid problems with ligatures)
    \let\@noligs\relax
%% DG/SR modification end
    \FV@Scan}
    \let\FVB@VerbatimOut\minted@FVB@VerbatimOut

\renewcommand\minted@savecode[1]{
  \immediate\openout\minted@code\jobname.pyg
  \immediate\write\minted@code{\expandafter\detokenize\expandafter{#1}}%
  \immediate\closeout\minted@code}
\makeatother


\title{- X -\linebreak Movimiento (III)}
\author{U-tad\\ Diseño Visual de Contenidos Digitales\\ Prof. Álvaro Castro-Castilla}
\date{}

\begin{document}
\maketitle


\section{Conceptos que emplearemos}
\begin{enumerate}
  \item Origen y ejes de coordenadas de dibujo

    \begin{minted}[linenos=true,bgcolor=bg]{java}
int x = 0;
int y = 0;

void setup() {
  size(500,500); 
}

void draw() {
  background(0);
  
  // Dibujar respecto al nuevo origen (x, y)
  translate(x, y);
  
  /* Dibuja la elipse en el punto (50, 50), sobre
     el eje de coordenadas habitual */
  ellipse(50, 50, 60, 60);
  
  /* Modificar x e y en cada fotograma
     (produce el efecto de movimiento) */
  x++; y++;
}
    \end{minted}

\newpage
  \item Múltiples orígenes y ejes de coordenadas, combinados

    \begin{minted}[linenos=true,bgcolor=bg]{java}
void setup() {
  size(500,500); 
}

void draw() {
  background(0);
  /* Realizamos una traslación del origen de
     coordenadas a (100, 0), es decir 100 píxeles
     hacia la derecha */
  translate(100, 0);
  // Dibujamos en el origen de las coordenadas
  ellipse(0, 0, 80, 80);
  
  /* Realizamos una nueva traslación, 250 píxeles
     hacia abajo */
  translate(0, 250);
  /* ... volvemos a dibujar en el origen
     de las coordenadas, en rojo */
  fill(255, 0, 0);
  ellipse(0, 0, 80, 80);
}
    \end{minted}

  \item Múltiples orígenes y ejes de coordenadas, aislados

    \begin{minted}[linenos=true,bgcolor=bg]{java}
void setup() {
  size(500,500); 
}

void draw() {
  background(0);
  /* Antes de la traslación, guardamos el sistema
     de coordenadas actual */
  pushMatrix();
  translate(100, 0);
  ellipse(0, 0, 80, 80);
  /* Al finalizar de dibujar sobre ese eje de
     coordenadas, volvemos al anterior */
  popMatrix();
  
  /* La traslación ahora se realiza sobre el sistema
     de coordeandas original, desplazándose
     verticalmente */
  translate(0, 250);
  fill(255, 0, 0);
  ellipse(0, 0, 80, 80);
}
    \end{minted}

\end{enumerate}


\section{Elementos de código que emplearemos}
\begin{enumerate}
  \item Traslación

    \begin{minted}[linenos=true,bgcolor=bg]{java}
void setup() {
  size(500, 500);
}

void draw() {
  // Traslación del sistema de coordenadas
  translate(mouseX, mouseY);
  rect(0, 0, 50, 50);
}
    \end{minted}

  \item Rotación

    \begin{minted}[linenos=true,bgcolor=bg]{java}
float angle = 0.0;

void setup() {
  size(500, 500);
}

void draw() {
  /* Antes de rotar el sistema de coordenadas,
     centrarlo en el lienzo */ 
  translate(width/2, height/2);
  // Rotar el sistema de coordeandas
  rotate(angle);
  rect(-25, -25, 50, 50);
  angle += 0.1;
}
    \end{minted}

  \item Escalado

    \begin{minted}[linenos=true,bgcolor=bg]{java}
float sizeScale = 1.0;

void setup() {
  size(500, 500);
}

void draw() {
  translate(width/2, height/2);
  // Si la escala es mayor que 10, recomenzar
  if (sizeScale > 10) {
    sizeScale = 1.0;
  }
  // Escalar el sistema de coordeandas 
  scale(sizeScale);
  rect(-25, -25, 50, 50);
  sizeScale += 0.1;
}
    \end{minted}

\end{enumerate}

\end{document}
