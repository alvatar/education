\documentclass[a4paper,oneside]{article}
% Graphics
\usepackage{graphicx}
\usepackage{wrapfig}
% Font selection
\usepackage[utf8]{inputenc}
\usepackage{PTSansNarrow} 
\renewcommand*\familydefault{\sfdefault} %% Only if the base font of the document is to be sans serif
\usepackage[T1]{fontenc}
% Code highlighting
\usepackage{minted}
% Links
\usepackage[hidelinks]{hyperref}

% Background color definition
\definecolor{bg}{rgb}{0.95,0.95,0.95}

% Code line numbers configuration
\renewcommand{\theFancyVerbLine}{\sffamily
  \textcolor[rgb]{0.5,0.5,1.0}{\scriptsize
     \oldstylenums{\arabic{FancyVerbLine}}}}


\makeatletter
\newcommand{\minted@write@detok}[1]{%
  \immediate\write\FV@OutFile{\detokenize{#1}}}%

\newcommand{\minted@FVB@VerbatimOut}[1]{%
  \@bsphack
  \begingroup
    \FV@UseKeyValues
    \FV@DefineWhiteSpace
    \def\FV@Space{\space}%
    \FV@DefineTabOut
    %\def\FV@ProcessLine{\immediate\write\FV@OutFile}% %Old, non-Unicode version
    \let\FV@ProcessLine\minted@write@detok %Patch for Unicode
    \immediate\openout\FV@OutFile #1\relax
    \let\FV@FontScanPrep\relax
%% DG/SR modification begin - May. 18, 1998 (to avoid problems with ligatures)
    \let\@noligs\relax
%% DG/SR modification end
    \FV@Scan}
    \let\FVB@VerbatimOut\minted@FVB@VerbatimOut

\renewcommand\minted@savecode[1]{
  \immediate\openout\minted@code\jobname.pyg
  \immediate\write\minted@code{\expandafter\detokenize\expandafter{#1}}%
  \immediate\closeout\minted@code}
\makeatother


\title{- XI -\linebreak Funciones}
\author{U-tad\\ Diseño Visual de Contenidos Digitales\\ Prof. Álvaro Castro-Castilla}
\date{}

\begin{document}
\maketitle


\section{Conceptos que emplearemos}
\begin{enumerate}
  \item Función básica

    \begin{minted}[linenos=true,bgcolor=bg]{java}
// Definición de la función "drawRandomCircle"
void drawRandomCircle() {
  ellipse(random(width), random(height), 50, 50);
}

void setup() {
  size(500,500);
}

// Mostrar cuatro círculos en posiciones aleatorias
void draw() {
  frameRate(1);
  background(255);
  drawRandomCircle();
  drawRandomCircle();
  drawRandomCircle();
  drawRandomCircle();
}
    \end{minted}

\newpage
  \item Función con parámetros

    \begin{minted}[linenos=true,bgcolor=bg]{java}
// Definición de la función "drawRandomCircle"
void drawMegaRobot(int x, int y) {
  ellipse(x, y+100, 150, 150);
  ellipse(x, y, 50, 50);
  ellipse(x-10, y, 10, 10);
  ellipse(x+10, y, 10, 10);
  rect(x-40, y+60, 80, 10);
  rect(x-40, y+80, 80, 10);
  rect(x-40, y+100, 80, 10);
}

void setup() {
  size(800, 800);
}

// Mostrar cuatro círculos en posiciones aleatorias
void draw() {
  frameRate(1);
  background(255);
  drawMegaRobot(100, 100);
  drawMegaRobot(400, 100);
  drawMegaRobot(700, 100);

  drawMegaRobot(300, 300);
  drawMegaRobot(500, 300);

  drawMegaRobot(100, 500);
  drawMegaRobot(400, 500);
  drawMegaRobot(700, 500);
}
    \end{minted}

  \item Función que devuelve un valor

    \begin{minted}[linenos=true,bgcolor=bg]{java}
int square(int x) {
  int result = x * x;
  return result;
}

println("El cuadrado de 3 es : " + square(3));
    \end{minted}
      
\end{enumerate}

\end{document}
