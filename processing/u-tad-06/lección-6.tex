\documentclass[a4paper,oneside]{article}
% Graphics
\usepackage{graphicx}
\usepackage{wrapfig}
% Font selection
\usepackage[utf8]{inputenc}
\usepackage{PTSansNarrow} 
\renewcommand*\familydefault{\sfdefault} %% Only if the base font of the document is to be sans serif
\usepackage[T1]{fontenc}
% Code highlighting
\usepackage{minted}
% Links
\usepackage[hidelinks]{hyperref}

% Background color definition
\definecolor{bg}{rgb}{0.95,0.95,0.95}

% Code line numbers configuration
\renewcommand{\theFancyVerbLine}{\sffamily
  \textcolor[rgb]{0.5,0.5,1.0}{\scriptsize
     \oldstylenums{\arabic{FancyVerbLine}}}}


\makeatletter
\newcommand{\minted@write@detok}[1]{%
  \immediate\write\FV@OutFile{\detokenize{#1}}}%

\newcommand{\minted@FVB@VerbatimOut}[1]{%
  \@bsphack
  \begingroup
    \FV@UseKeyValues
    \FV@DefineWhiteSpace
    \def\FV@Space{\space}%
    \FV@DefineTabOut
    %\def\FV@ProcessLine{\immediate\write\FV@OutFile}% %Old, non-Unicode version
    \let\FV@ProcessLine\minted@write@detok %Patch for Unicode
    \immediate\openout\FV@OutFile #1\relax
    \let\FV@FontScanPrep\relax
%% DG/SR modification begin - May. 18, 1998 (to avoid problems with ligatures)
    \let\@noligs\relax
%% DG/SR modification end
    \FV@Scan}
    \let\FVB@VerbatimOut\minted@FVB@VerbatimOut

\renewcommand\minted@savecode[1]{
  \immediate\openout\minted@code\jobname.pyg
  \immediate\write\minted@code{\expandafter\detokenize\expandafter{#1}}%
  \immediate\closeout\minted@code}
\makeatother


\title{- VI -\linebreak Respuesta e interacción}
\author{U-tad\\ Diseño Visual de Contenidos Digitales\\ Prof. Álvaro Castro-Castilla}
\date{}

\begin{document}
\maketitle


\section{Conceptos que emplearemos}
\begin{enumerate}
  \item Comprobación del estado del ratón

    \begin{minted}[linenos=true,bgcolor=bg]{java}
void setup() {
  smooth();
  strokeWeight(30);
}
void draw() {
  if (mousePressed) { // (Igual que mousePressed == true)
    stroke(0);
  } else {
    stroke(102);
  }
  ellipse(mouseX - 20, mouseY - 20, 40, 40);
}
    \end{minted}

  \item Funciones de evento

    \begin{minted}[linenos=true,bgcolor=bg]{java}
void mousePressed() {
  fill(255);
}

void mouseReleased() {
  fill(0);
}
    \end{minted}

\end{enumerate}

\pagebreak

\section{Elementos de código que emplearemos}
\begin{enumerate}
  \item Detección del botón pulsado

    \begin{minted}[linenos=true,bgcolor=bg]{java}
if (mousePressed) {
  if (mouseButton == LEFT) {
    stroke(0);
  } else {
    stroke(255);
  }
}
    \end{minted}

  \item Pulsación de teclas

    \begin{minted}[linenos=true,bgcolor=bg]{java}
int robotPositionX = 0;
int robotPositionY = 0;

if (keyPressed) {
  if ((key == 'a') || (key == 'A')) {
    robotPositionX -= 2;
  } else if ((key == 'd') || (key == 'D')) {
    robotPositionX += 2;
  } else if ((key == 'w') || (key == 'W')) {
    robotPositionY += 2;
  } else if ((key == 's') || (key == 'S')) {
    robotPositionY -= 2;
  }
}
    \end{minted}
      
  \item Detección de teclas mediante \textbf{keycodes}

    \begin{minted}[linenos=true,bgcolor=bg]{java}
int robotPositionX = 0;
int robotPositionY = 0;

if (keyPressed) {
  if (keyCode == LEFT) {
    robotPositionX -= 2;
  } else if (keyCode == RIGHT) {
    robotPositionX += 2;
  } else if (keyCode == UP) {
    robotPositionY += 2;
  } else if (keyCode == DOWN) {
    robotPositionY -= 2;
  }
}
    \end{minted}

\end{enumerate}

\end{document}
