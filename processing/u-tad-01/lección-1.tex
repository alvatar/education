\documentclass[14pt,a4paper,oneside]{article}
% Graphics
\usepackage{graphicx}
\usepackage{wrapfig}
% Font selection
\usepackage[utf8]{inputenc}
\usepackage{PTSansNarrow} 
\renewcommand*\familydefault{\sfdefault} %% Only if the base font of the document is to be sans serif
\usepackage[T1]{fontenc}
% Code highlighting
\usepackage{minted}
% Links
\usepackage[hidelinks]{hyperref}

\definecolor{bg}{rgb}{0.95,0.95,0.95}

\renewcommand{\theFancyVerbLine}{\sffamily
\textcolor[rgb]{0.5,0.5,1.0}{\scriptsize
\oldstylenums{\arabic{FancyVerbLine}}}}


\title{- I -\linebreak El lienzo de dibujo y las formas básicas}
\author{}
\date{}

\begin{document}
\maketitle

\section{Tutoriales online relacionados}

\begin{enumerate}
   \item \href{http://processing.org/learning/drawing/}{Sistema de coordenadas y formas} 
   \item \href{http://go.yuri.at/p5/tutorial/}{Tutorial básico en castellano}
\end{enumerate}

\section{Bloques básicos de código que emplearemos}


\section{Ejemplos}

\begin{minted}[linenos=true,bgcolor=bg]{c}
int main() {
  printf("hello, world");
  return 0;
}
\end{minted}

\end{document}
