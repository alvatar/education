\documentclass[a4paper,oneside]{article}
% Graphics
\usepackage{graphicx}
\usepackage{wrapfig}
% Font selection
\usepackage[utf8]{inputenc}
\usepackage{PTSansNarrow} 
\renewcommand*\familydefault{\sfdefault} %% Only if the base font of the document is to be sans serif
\usepackage[T1]{fontenc}
% Code highlighting
\usepackage{minted}
% Links
\usepackage[hidelinks]{hyperref}

% Background color definition
\definecolor{bg}{rgb}{0.95,0.95,0.95}

% Code line numbers configuration
\renewcommand{\theFancyVerbLine}{\sffamily
  \textcolor[rgb]{0.5,0.5,1.0}{\scriptsize
     \oldstylenums{\arabic{FancyVerbLine}}}}


\makeatletter
\newcommand{\minted@write@detok}[1]{%
  \immediate\write\FV@OutFile{\detokenize{#1}}}%

\newcommand{\minted@FVB@VerbatimOut}[1]{%
  \@bsphack
  \begingroup
    \FV@UseKeyValues
    \FV@DefineWhiteSpace
    \def\FV@Space{\space}%
    \FV@DefineTabOut
    %\def\FV@ProcessLine{\immediate\write\FV@OutFile}% %Old, non-Unicode version
    \let\FV@ProcessLine\minted@write@detok %Patch for Unicode
    \immediate\openout\FV@OutFile #1\relax
    \let\FV@FontScanPrep\relax
%% DG/SR modification begin - May. 18, 1998 (to avoid problems with ligatures)
    \let\@noligs\relax
%% DG/SR modification end
    \FV@Scan}
    \let\FVB@VerbatimOut\minted@FVB@VerbatimOut

\renewcommand\minted@savecode[1]{
  \immediate\openout\minted@code\jobname.pyg
  \immediate\write\minted@code{\expandafter\detokenize\expandafter{#1}}%
  \immediate\closeout\minted@code}
\makeatother


\title{- I -\linebreak El lienzo de dibujo y las formas básicas}
\author{U-tad\\ Diseño Visual de Contenidos Digitales\\ Prof. Álvaro Castro-Castilla}
\date{}

\begin{document}
\maketitle


\section{Tutoriales online relacionados}
\begin{enumerate}
  \item \href{http://processing.org/learning/drawing/}{Sistema de coordenadas y formas} 
  \item \href{http://go.yuri.at/p5/tutorial/}{Tutorial básico en castellano}
\end{enumerate}


\section{Conceptos que emplearemos}
\begin{enumerate}
  \item Bloque de código de inicialización: \textbf{setup()}

    \begin{minted}[linenos=true,bgcolor=bg]{java}
void setup() {
  // Aquí escribiremos nuestro código de inicialización
}
    \end{minted}
  \item Bloque de código de dibujo: \textbf{draw()}

    \begin{minted}[linenos=true,bgcolor=bg]{java}
void draw() {
  // Aquí escribiremos nuestro código de dibujo
}
    \end{minted}
  \item Orden de dibujado

    \begin{minted}[linenos=true,bgcolor=bg]{java}
void draw() {
  // Se dibuja primero
  rect(10, 10, 80, 80);
  // Se dibuja después, superponiéndose
  rect(20, 20, 90, 90);
}
    \end{minted}
\end{enumerate}


\section{Elementos de código que emplearemos}
\begin{enumerate}
  \item Comentarios del código

    \begin{minted}[linenos=true,bgcolor=bg]{java}
// comentario de una linea

/* Comentarios de varias líneas. Una
   alineación correcta ayuda a entender
   el código y a leer el comentario */
    \end{minted}
  \item \href{http://processing.org/reference/}{Link de referencia oficial de las funciones}
  \item Dibujar un punto

    \mint[bgcolor=bg]{java} |point(...);|
  \item Dibujar un rectángulo

    \mint[bgcolor=bg]{java} |rect(...);|
  \item Dibujar una elipse o círculo

    \mint[bgcolor=bg]{java} |ellipse(...);|
  \item Dibujar una línea

    \mint[bgcolor=bg]{java} |line(...);|
  \item Dibujar un triángulo

    \mint[bgcolor=bg]{java} |triangle(...);|
  \item Dibujar un cuadrilátero

    \mint[bgcolor=bg]{java} |quad(...);|
  \item Dibujar un arco de círculo

    \mint[bgcolor=bg]{java} |arc(...);|
\end{enumerate}


\section{Ejemplos}
\begin{enumerate}
  \item Dibujo estático directo sobre el lienzo

    \begin{minted}[linenos=true,bgcolor=bg]{java}
rect(50, 50, 90, 100);
rect(70, 20, 100, 120);
    \end{minted}
  \item Animación interactiva de nuestro dibujo

    \begin{minted}[linenos=true,bgcolor=bg]{java}
// Inicialización del lienzo
void setup() {
  // Establecer el tamaño del lienzo
  size(400, 400);
}

// Dibujo interactivo sobre el lienzo
void draw() {
  /* La posición del rectángulo depenede de la posición
     del ratón en cada momento mouseX y mouseY */
  rect(mouseX, mouseY, 80, 80);
}
    \end{minted}
\end{enumerate}

\end{document}
