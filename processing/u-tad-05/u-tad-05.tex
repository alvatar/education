\documentclass[a4paper,oneside]{article}
% Graphics
\usepackage{graphicx}
\usepackage{wrapfig}
% Font selection
\usepackage[utf8]{inputenc}
\usepackage{PTSansNarrow} 
\renewcommand*\familydefault{\sfdefault} %% Only if the base font of the document is to be sans serif
\usepackage[T1]{fontenc}
% Code highlighting
\usepackage{minted}
% Links
\usepackage[hidelinks]{hyperref}

% Background color definition
\definecolor{bg}{rgb}{0.95,0.95,0.95}

% Code line numbers configuration
\renewcommand{\theFancyVerbLine}{\sffamily
  \textcolor[rgb]{0.5,0.5,1.0}{\scriptsize
     \oldstylenums{\arabic{FancyVerbLine}}}}


\makeatletter
\newcommand{\minted@write@detok}[1]{%
  \immediate\write\FV@OutFile{\detokenize{#1}}}%

\newcommand{\minted@FVB@VerbatimOut}[1]{%
  \@bsphack
  \begingroup
    \FV@UseKeyValues
    \FV@DefineWhiteSpace
    \def\FV@Space{\space}%
    \FV@DefineTabOut
    %\def\FV@ProcessLine{\immediate\write\FV@OutFile}% %Old, non-Unicode version
    \let\FV@ProcessLine\minted@write@detok %Patch for Unicode
    \immediate\openout\FV@OutFile #1\relax
    \let\FV@FontScanPrep\relax
%% DG/SR modification begin - May. 18, 1998 (to avoid problems with ligatures)
    \let\@noligs\relax
%% DG/SR modification end
    \FV@Scan}
    \let\FVB@VerbatimOut\minted@FVB@VerbatimOut

\renewcommand\minted@savecode[1]{
  \immediate\openout\minted@code\jobname.pyg
  \immediate\write\minted@code{\expandafter\detokenize\expandafter{#1}}%
  \immediate\closeout\minted@code}
\makeatother


\title{- V -\linebreak Condiciones}
\author{U-tad\\ Diseño Visual de Contenidos Digitales\\ Prof. Álvaro Castro-Castilla}
\date{}

\begin{document}
\maketitle


\section{Conceptos que emplearemos}
\begin{enumerate}
  \item If

    \begin{minted}[linenos=true,bgcolor=bg]{java}
if (mouseX > 200) {
  println("mouseX es mayor que 200");
}
    \end{minted}

  \item If ... else

    \begin{minted}[linenos=true,bgcolor=bg]{java}
if (mouseX > 200) {
  println("mouseX es mayor que 200");
} else {
  println("mouseX NO es mayor que 200");
}
    \end{minted}

  \item If ... else if ... else

    \begin{minted}[linenos=true,bgcolor=bg]{java}
if (mouseX > 200) {
  println("mouseX es mayor que 200");
} else if (mouseX > 100) {
  println("mouseX es mayor que 100 (y menor que 200)");
} else {
  println("mouseX NO es mayor que 100 ni que 200");
}
    \end{minted}
      
\end{enumerate}


\section{Elementos de código que emplearemos}
\begin{enumerate}
  \item Operaciones booleanas

    \begin{minted}[linenos=true,bgcolor=bg]{java}
// OR (... o ...)
||

// AND (... y ...)
&&

// NOT (lo contrario de...)
!
    \end{minted}
\end{enumerate}

\end{document}
