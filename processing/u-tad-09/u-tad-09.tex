\documentclass[a4paper,oneside]{article}
% Graphics
\usepackage{graphicx}
\usepackage{wrapfig}
% Font selection
\usepackage[utf8]{inputenc}
\usepackage{PTSansNarrow} 
\renewcommand*\familydefault{\sfdefault} %% Only if the base font of the document is to be sans serif
\usepackage[T1]{fontenc}
% Code highlighting
\usepackage{minted}
% Links
\usepackage[hidelinks]{hyperref}

% Background color definition
\definecolor{bg}{rgb}{0.95,0.95,0.95}

% Code line numbers configuration
\renewcommand{\theFancyVerbLine}{\sffamily
  \textcolor[rgb]{0.5,0.5,1.0}{\scriptsize
     \oldstylenums{\arabic{FancyVerbLine}}}}


\makeatletter
\newcommand{\minted@write@detok}[1]{%
  \immediate\write\FV@OutFile{\detokenize{#1}}}%

\newcommand{\minted@FVB@VerbatimOut}[1]{%
  \@bsphack
  \begingroup
    \FV@UseKeyValues
    \FV@DefineWhiteSpace
    \def\FV@Space{\space}%
    \FV@DefineTabOut
    %\def\FV@ProcessLine{\immediate\write\FV@OutFile}% %Old, non-Unicode version
    \let\FV@ProcessLine\minted@write@detok %Patch for Unicode
    \immediate\openout\FV@OutFile #1\relax
    \let\FV@FontScanPrep\relax
%% DG/SR modification begin - May. 18, 1998 (to avoid problems with ligatures)
    \let\@noligs\relax
%% DG/SR modification end
    \FV@Scan}
    \let\FVB@VerbatimOut\minted@FVB@VerbatimOut

\renewcommand\minted@savecode[1]{
  \immediate\openout\minted@code\jobname.pyg
  \immediate\write\minted@code{\expandafter\detokenize\expandafter{#1}}%
  \immediate\closeout\minted@code}
\makeatother


\title{- IX -\linebreak Movimiento (II)}
\author{U-tad\\ Diseño Visual de Contenidos Digitales\\ Prof. Álvaro Castro-Castilla}
\date{}

\begin{document}
\maketitle


\section{Conceptos que emplearemos}
\begin{enumerate}
  \item Movimiento ondular

    \begin{minted}[linenos=true,bgcolor=bg]{java}
// Ángulo
float angle = 0.0;
// Desplazamiento inicial
float offset = 200;
// Amplitud del movimiento ondular
float amplitude = 40;
// Velocidad del movimiento ondular
float speed = 0.05;

void setup() {
  size(500, 500);
}

void draw() {
  background(0);

  // Ecuación del movimiento ondular
  float y = offset + sin(angle) * amplitude;

  ellipse( 200, y, 40, 40);

  /* Sumamos la velocidad al ángulo, en lugar de a
     la posición */
  angle += speed;
}
    \end{minted}

\newpage
  \item Movimiento circular

    \begin{minted}[linenos=true,bgcolor=bg]{java}
// Ángulo
float angle = 0.0;
// Desplazamientos del origen del movimiento
float offsetX = 150;
float offsetY = 250;
// Amplitud del movimiento
float amplitude = 60;
// Velocidad del movimiento
float speed = 0.05;

void setup() {
  size(500, 500);
  smooth();
}

void draw() {
  // Ecuaciones del movimiento circular
  float x = offsetX + cos(angle) * amplitude;
  float y = offsetY + sin(angle) * amplitude;

  ellipse( x, y, 40, 40);

  // Sumamos la velocidad al ángulo
  angle += speed;
}
    \end{minted}

\end{enumerate}

\end{document}
