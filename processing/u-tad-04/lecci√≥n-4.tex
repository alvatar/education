\documentclass[a4paper,oneside]{article}
% Graphics
\usepackage{graphicx}
\usepackage{wrapfig}
% Font selection
\usepackage[utf8]{inputenc}
\usepackage{PTSansNarrow} 
\renewcommand*\familydefault{\sfdefault} %% Only if the base font of the document is to be sans serif
\usepackage[T1]{fontenc}
% Code highlighting
\usepackage{minted}
% Links
\usepackage[hidelinks]{hyperref}

% Background color definition
\definecolor{bg}{rgb}{0.95,0.95,0.95}

% Code line numbers configuration
\renewcommand{\theFancyVerbLine}{\sffamily
  \textcolor[rgb]{0.5,0.5,1.0}{\scriptsize
     \oldstylenums{\arabic{FancyVerbLine}}}}


\makeatletter
\newcommand{\minted@write@detok}[1]{%
  \immediate\write\FV@OutFile{\detokenize{#1}}}%

\newcommand{\minted@FVB@VerbatimOut}[1]{%
  \@bsphack
  \begingroup
    \FV@UseKeyValues
    \FV@DefineWhiteSpace
    \def\FV@Space{\space}%
    \FV@DefineTabOut
    %\def\FV@ProcessLine{\immediate\write\FV@OutFile}% %Old, non-Unicode version
    \let\FV@ProcessLine\minted@write@detok %Patch for Unicode
    \immediate\openout\FV@OutFile #1\relax
    \let\FV@FontScanPrep\relax
%% DG/SR modification begin - May. 18, 1998 (to avoid problems with ligatures)
    \let\@noligs\relax
%% DG/SR modification end
    \FV@Scan}
    \let\FVB@VerbatimOut\minted@FVB@VerbatimOut

\renewcommand\minted@savecode[1]{
  \immediate\openout\minted@code\jobname.pyg
  \immediate\write\minted@code{\expandafter\detokenize\expandafter{#1}}%
  \immediate\closeout\minted@code}
\makeatother


\title{- IV -\linebreak Repetición (bucles)}
\author{U-tad\\ Diseño Visual de Contenidos Digitales\\ Prof. Álvaro Castro-Castilla}
\date{}

\begin{document}
\maketitle


\section{Conceptos que emplearemos}
\begin{enumerate}
  \item Bucles \textbf{for}

    \begin{minted}[linenos=true,bgcolor=bg]{java}
      
/* Estructura básica de un bucle "for"
 *
 * 1) Inicialización de la variable i
 * 2) Test de la variable i (< 10)
 * 3) Incremento de i
 * 4) Código a ejecutar
 * 
 * for (<1>inicialización; <2>test; <3>modificación) {
 *   <4>instrucciones a repetir
 * }
 *
 */
for (int i = 10; i < 200; i += 10) {
  // El siguiente comando se repetirá para
  // cada valor de i
  line(i + 20, 20, i + 30, 50);
}

/* Típico bucle, que comienza con i = 0
 * y se incrementa de uno en uno
 */
for (int i = 0; i < 20; i++) {
  // Salida del valor de i por la consola
  println(i);
}
    \end{minted}

  \item Bucles dentro de bucles

    \begin{minted}[linenos=true,bgcolor=bg]{java}
for (int y = 0; y <= 200; y += 20) {
  for (int x = 0; x <= 200; x += 20) {
    ellipse(x, y, 10, 10);
  }
}
    \end{minted}
      
\end{enumerate}


\section{Elementos de código que emplearemos}
\begin{enumerate}
  \item Tests

    \begin{minted}[linenos=true,bgcolor=bg]{java}
// Igual que ...
==

// No igual que ...
!=

// Mayor que ...
> 

// Menor que ...
< 

// Mayor o igual que ...
>=  

// Menor o igual que ...
<=  

// Las emplearemos en la parte (2) de un bucle for:
for (int = 0; /* ¡Aquí! -> */ i != 10; i++) {
  println(i);
}
    \end{minted}
  \item Salida por consola (imprimir un dato por pantalla)

    \mint[bgcolor=bg]{java} |println(mi_variable);|
\end{enumerate}

\end{document}
