\documentclass[a4paper,oneside]{article}
% Graphics
\usepackage{graphicx}
\usepackage{wrapfig}
% Font selection
\usepackage[utf8]{inputenc}
\usepackage{PTSansNarrow} 
\renewcommand*\familydefault{\sfdefault} %% Only if the base font of the document is to be sans serif
\usepackage[T1]{fontenc}
% Code highlighting
\usepackage{minted}
% Links
\usepackage[hidelinks]{hyperref}

% Background color definition
\definecolor{bg}{rgb}{0.95,0.95,0.95}

% Code line numbers configuration
\renewcommand{\theFancyVerbLine}{\sffamily
  \textcolor[rgb]{0.5,0.5,1.0}{\scriptsize
     \oldstylenums{\arabic{FancyVerbLine}}}}


\makeatletter
\newcommand{\minted@write@detok}[1]{%
  \immediate\write\FV@OutFile{\detokenize{#1}}}%

\newcommand{\minted@FVB@VerbatimOut}[1]{%
  \@bsphack
  \begingroup
    \FV@UseKeyValues
    \FV@DefineWhiteSpace
    \def\FV@Space{\space}%
    \FV@DefineTabOut
    %\def\FV@ProcessLine{\immediate\write\FV@OutFile}% %Old, non-Unicode version
    \let\FV@ProcessLine\minted@write@detok %Patch for Unicode
    \immediate\openout\FV@OutFile #1\relax
    \let\FV@FontScanPrep\relax
%% DG/SR modification begin - May. 18, 1998 (to avoid problems with ligatures)
    \let\@noligs\relax
%% DG/SR modification end
    \FV@Scan}
    \let\FVB@VerbatimOut\minted@FVB@VerbatimOut

\renewcommand\minted@savecode[1]{
  \immediate\openout\minted@code\jobname.pyg
  \immediate\write\minted@code{\expandafter\detokenize\expandafter{#1}}%
  \immediate\closeout\minted@code}
\makeatother


\title{- VII -\linebreak Imágenes, tipografías y gráficos vectoriales}
\author{U-tad\\ Diseño Visual de Contenidos Digitales\\ Prof. Álvaro Castro-Castilla}
\date{}

\begin{document}
\maketitle


\section{Elementos de código que emplearemos}
\begin{enumerate}
  \item Emplear imágenes

    \begin{minted}[linenos=true,bgcolor=bg]{java}
// La variable debe ser global
PImage img;

// Cargar la imagen (sólo se hace una vez)
void setup() {
  size(500, 500);
  img = loadImage("archivo-imagen.jpg");
}

// Mostrar la imagen
void draw() {
  background(0);
  image(img, 0, 0);
}
    \end{minted}

  \item Mostrar texto y emplear fuentes

    \begin{minted}[linenos=true,bgcolor=bg]{java}
// La variable debe ser global
PFont font;

// Cargar y preparar las fuentes
void setup() {
  size(500, 500);
  font = loadFont("AndaleMono-36.vlw");
  textFont(font);
}

// Mostrar texto con esta fuente
void draw() {
  background(100);
  textSize(36);
  text("!Hola mundo!", 100, 100);
}
    \end{minted}
      
  \item Strings

    \begin{minted}[linenos=true,bgcolor=bg]{java}
// Los Strings o cadenas de caracteres son el tipo de variable
// que nos permitirán guardar texto
String myText = "Hola mundo";
// ...para posteriormente usarlo y/o modificarlo
text( myText + "!" );
    \end{minted}

  \item Formas vectoriales diseñadas con Illustrator o Photoshop

    \begin{minted}[linenos=true,bgcolor=bg]{java}
PShape myDrawing;

void setup() {
  size(480, 120);
  smooth();
  myDrawing = loadShape("archivo_vectorial.svg");
}

void draw() {
  background(0);
  shape(myDrawing, mouseX, mouseY);
}
    \end{minted}

\end{enumerate}

\end{document}
