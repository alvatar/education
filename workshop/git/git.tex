\documentclass[a4paper,oneside]{article}
% Graphics
\usepackage{graphicx}
\usepackage{wrapfig}
% Font selection
\usepackage[utf8]{inputenc}
\usepackage{PTSansNarrow} 
\renewcommand*\familydefault{\sfdefault} %% Only if the base font of the document is to be sans serif
\usepackage[T1]{fontenc}
% Code highlighting
\usepackage{minted}
% Links
\usepackage[hidelinks]{hyperref}

% Background color definition
\definecolor{bg}{rgb}{0.95,0.95,0.95}

% Code line numbers configuration
\renewcommand{\theFancyVerbLine}{\sffamily
  \textcolor[rgb]{0.5,0.5,1.0}{\scriptsize
     \oldstylenums{\arabic{FancyVerbLine}}}}


\makeatletter
\newcommand{\minted@write@detok}[1]{%
  \immediate\write\FV@OutFile{\detokenize{#1}}}%

\newcommand{\minted@FVB@VerbatimOut}[1]{%
  \@bsphack
  \begingroup
    \FV@UseKeyValues
    \FV@DefineWhiteSpace
    \def\FV@Space{\space}%
    \FV@DefineTabOut
    %\def\FV@ProcessLine{\immediate\write\FV@OutFile}% %Old, non-Unicode version
    \let\FV@ProcessLine\minted@write@detok %Patch for Unicode
    \immediate\openout\FV@OutFile #1\relax
    \let\FV@FontScanPrep\relax
%% DG/SR modification begin - May. 18, 1998 (to avoid problems with ligatures)
    \let\@noligs\relax
%% DG/SR modification end
    \FV@Scan}
    \let\FVB@VerbatimOut\minted@FVB@VerbatimOut

\renewcommand\minted@savecode[1]{
  \immediate\openout\minted@code\jobname.pyg
  \immediate\write\minted@code{\expandafter\detokenize\expandafter{#1}}%
  \immediate\closeout\minted@code}
\makeatother


\title{Referencia de uso de Git}
\author{U-tad\\ Ingeniería en Desarrollo de Contenidos Digitales\\ Prof. Álvaro Castro-Castilla}
\date{}

\begin{document}
\maketitle


\section{Configuración inicial}
\begin{enumerate}
  \item Nombre y dirección de correo (sólo necesario al instalar Git por primera vez)

    \begin{minted}[linenos=true,bgcolor=bg]{bash}
$ git config --global user.name "tu nombre"
$ git config --global user.email "tu email"
    \end{minted}

  \item Inicialización de un repositorio

Dentro de la carpeta que hemos creado para nuestro proyecto, ejecutar:

    \begin{minted}[linenos=true,bgcolor=bg]{bash}
$ git init
    \end{minted}
\end{enumerate}


\section{Uso básico de un repositorio Git}
\begin{enumerate}

  \item Marcar archivos para que Git registre sus cambios:

    \begin{minted}[linenos=true,bgcolor=bg]{bash}
$ git add <mi-archivo>
    \end{minted}

Marcar todos los archivos de la carpeta

    \begin{minted}[linenos=true,bgcolor=bg]{bash}
$ git add .
    \end{minted}

Para registrar todos los cambios, incluyendo las eliminaciones de archivos, se recomiendo usar siempre el \textbf{flag -A}:

    \begin{minted}[linenos=true,bgcolor=bg]{bash}
$ git add -A <mi-archivo>
    \end{minted}

  \item Archivo \textbf{.gitignore} para que Git no pueda marcar algunos archivos

El archivo .gitignore contendrá un listado de archivos que no queremos que sean registrados por Git. Por ejemplo:

    \begin{minted}[linenos=true,bgcolor=bg]{bash}
$ cat .gitignore

.DS_Store
*.db
node_modules/
    \end{minted}

  \item Comprobar el estado del repositorio (y ver si tenemos algo pendiente de marcar o actualizar en Git)

Este comando nos informará si tenemos algún archivo por actulizar, o pendiente de ingresar en el repositorio:

    \begin{minted}[linenos=true,bgcolor=bg]{bash}
$ git status
    \end{minted}

El resultado mostrará listas de archivos con la siguiente información:
    \begin{enumerate}
      \item \# Untracked files:

        Estos archivos no han sido marcados por Git todavía para poder seguir sus cambios. Usar \textbf{git add}
      \item \# Changes to be committed:

        Estos archivos han sido resgistrados para introducir en la base de datos de cambios de Git, pero están a la espera de realizarlo. Usar para ello \textbf{git commit -m <descripción de commit>}
      \item \# Changes not staged for commit:

        Estos archivos ya han sido registrados en Git anteriormente, pero han sufrido cambios desde la última actualización en la base de datos de cambios de Git. Usar \textbf{git add <archivo o .>} para registrar el archivo.
    \end{enumerate}

  \item Ingresar los cambios en la base de datos

    \begin{minted}[linenos=true,bgcolor=bg]{bash}
$ git commit -m <mensaje descriptivo del cambio>
    \end{minted}

  \item Visualizar los cambios registrados en Git

    \begin{minted}[linenos=true,bgcolor=bg]{bash}
$ git log
$ git log --pretty=oneline
    \end{minted}



\end{enumerate}

\end{document}
