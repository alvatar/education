\documentclass[a4paper,oneside]{article}
% Graphics
\usepackage{graphicx}
\usepackage{wrapfig}
% Font selection
\usepackage[utf8]{inputenc}
\usepackage{PTSansNarrow} 
\renewcommand*\familydefault{\sfdefault} %% Only if the base font of the document is to be sans serif
\usepackage[T1]{fontenc}
% Code highlighting
\usepackage{minted}
% Links
\usepackage[hidelinks]{hyperref}

% Background color definition
\definecolor{bg}{rgb}{0.95,0.95,0.95}

% Code line numbers configuration
\renewcommand{\theFancyVerbLine}{\sffamily
  \textcolor[rgb]{0.5,0.5,1.0}{\scriptsize
     \oldstylenums{\arabic{FancyVerbLine}}}}


\makeatletter
\newcommand{\minted@write@detok}[1]{%
  \immediate\write\FV@OutFile{\detokenize{#1}}}%

\newcommand{\minted@FVB@VerbatimOut}[1]{%
  \@bsphack
  \begingroup
    \FV@UseKeyValues
    \FV@DefineWhiteSpace
    \def\FV@Space{\space}%
    \FV@DefineTabOut
    %\def\FV@ProcessLine{\immediate\write\FV@OutFile}% %Old, non-Unicode version
    \let\FV@ProcessLine\minted@write@detok %Patch for Unicode
    \immediate\openout\FV@OutFile #1\relax
    \let\FV@FontScanPrep\relax
%% DG/SR modification begin - May. 18, 1998 (to avoid problems with ligatures)
    \let\@noligs\relax
%% DG/SR modification end
    \FV@Scan}
    \let\FVB@VerbatimOut\minted@FVB@VerbatimOut

\renewcommand\minted@savecode[1]{
  \immediate\openout\minted@code\jobname.pyg
  \immediate\write\minted@code{\expandafter\detokenize\expandafter{#1}}%
  \immediate\closeout\minted@code}
\makeatother


\title{Conceptos fundamentales de Scheme}
\author{U-tad\\ Ingeniería en Desarollo de Contenidos Digitales\\ Prof. Álvaro Castro-Castilla}
\date{}

\begin{document}
\maketitle


\section{Estructuras básicas}
\begin{enumerate}
  \item Crear variables

    \begin{minted}[linenos=true,bgcolor=bg]{scheme}
(define my-variable 'value)
    \end{minted}

  \item Crear funciones

    \begin{minted}[linenos=true,bgcolor=bg]{scheme}
;; Opción 1, igual que una variable pero con lambda
(define my-function
  (lambda (parameter1 parameter2)
    (+ parameter1 parameter2)))

;; Opción 2 (sintaxis reducida)
(define (my-function parameter1 parameter2)
  (+ parameter1 parameter2))
    \end{minted}

\end{enumerate}

\section{Conceptos básicos}
\begin{enumerate}
  \item Principio de sustitución

    \begin{minted}[linenos=true,bgcolor=bg]{scheme}
(+ 2 2)
;; Equivale a su resultado de computar la expresión:
4
;; Es decir, se sustituye la expresión por su resultado

(+ 3 (* 5 7) (- 7 2))
;; ...todo ello se puede sustituir por:
43
    \end{minted}

\pagebreak
  \item Devolución de valores

    \begin{minted}[linenos=true,bgcolor=bg]{scheme}
(define (my-func a b)
  (+ a b)
  (* a b))
(my-func 4 5)       ===> 20
;; Siguiendo el principio de sustitución, el resultado será
;; 20 descartándose el resultado de la expresión anterior,
;; ya que (* a b) es lo último que queda al ejecutar el
;; código dentro de la función

;; Otro ejemplo:
(begin
 (println "Hola ")
 (println "mundo")
 (string-append "Adiós " "mundo"))
;; Este código ejecutará las tres instrucciónes, pero el
;; resultado que la función devolverá será el resultado de
;; computar la última expresión ===> "Adiós mundo"
    \end{minted}

\end{enumerate}

\section{Estructuras de control}
\begin{enumerate}
  \item if

    \begin{minted}[linenos=true,bgcolor=bg]{scheme}
(if (null? l) ; Test
    (println "list is null") ; si test es true
    (println "list is NOT null")) ; si test es false
    \end{minted}

  \item when (if + begin)

    \begin{minted}[linenos=true,bgcolor=bg]{scheme}
;; When equivale a un if en el que sólo se considera el
;; caso "true", ejecutándose todas las instrucciones a
;; continuación. En caso de que el test de #f, unless
;; devuelve esto
(when (test) ; Test
      (println "It was true") ; Expresión 1
      (println "so all this is executed") ; Expresión 2...
      'but-this-is-returned) ; Se devuelve este resultado
    \end{minted}

  \item unless (if-not + begin)

    \begin{minted}[linenos=true,bgcolor=bg]{scheme}
;; When equivale a un if en el que sólo se considera el
;; caso "false", ejecutándose todas las instrucciones a
;; continuación. En caso contrario, es decir que el test
;; resulte en "true", unless devuelve #f
(unless (test) ; Test
        (println "It was false") ; Expresión 1
        (println "so all this is executed") ; Expresión 2...
        'but-this-is-returned) ; Se devuelve este resultado
    \end{minted}

  \item cond

    \begin{minted}[linenos=true,bgcolor=bg]{scheme}
(cond
  ((null? l) ; test 1
   (println "list is null")) ; si test 1 es true
  ((eq? (car l) 'a) ; test 2
   (println "first element is 'a")) ; si test 2 es true
  (else ; si no se cumple ninguna de las anteriores
   (println "none of the clauses were met")))
    \end{minted}

  \item case

    \begin{minted}[linenos=true,bgcolor=bg]{scheme}
(case (get-a-symbol)
  ((a b) ; ¿es el resultado de (get-a-symbol) 'a o 'b?
   (println "a or b")) ; si es 'a o 'b
  ((c) ; ¿es el resultado de (get-a-symbol) 'c?
   (println "first element is 'c")) ; si es 'c
  (else ; si no se cumple ninguna de las anteriores
   (println "none of the clauses were met")))
    \end{minted}

\end{enumerate}

\pagebreak
\section{Objetos}
\begin{enumerate}
  \item Definition de un tipo

    \begin{minted}[linenos=true,bgcolor=bg]{scheme}
;; El tipo "robot" tendrá 3 campos o atributos:
;;   - head
;;   - arms
;;   - legs
(define-structure my-robot head arms legs)
    \end{minted}

  \item Creación de una instancia MAKE-[TIPO]

    \begin{minted}[linenos=true,bgcolor=bg]{scheme}
;; Crear un robot, en la variable que se llame "my-robot"
;; y hacer que head, arms y legs tengan valores:
(define my-robot
  (make-robot 'big-head 'short-arms 'mechanic-legs))
    \end{minted}

  \item Predicado (test: es esto un robot?) [TIPO]?

    \begin{minted}[linenos=true,bgcolor=bg]{scheme}
;; Comprobar si los siguientes valores o variables son
;; robots
(robot? my-robot)   ===> #t
(robot? 1234)       ===> #f
    \end{minted}

  \item Acceso a los campos o atributos [TIPO]-[CAMPO]

    \begin{minted}[linenos=true,bgcolor=bg]{scheme}
;; Extraer o acceder al campo "arms" de la instancia
;; "my-robot"
(robot-arms my-robot)   ===> 'short-arms
    \end{minted}

  \item Modificación de campos [TIPO]-[CAMPO]-SET!

    \begin{minted}[linenos=true,bgcolor=bg]{scheme}
;; Modificar el campo "arms" de "my-robot" con el valor
;; 'long-arms
(robot-arms-set! robot 'long-arms)
    \end{minted}

\end{enumerate}

\end{document}
