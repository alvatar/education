\documentclass[a4paper,oneside]{article}
% Graphics
\usepackage{graphicx}
\usepackage{wrapfig}
% Font selection
\usepackage[utf8]{inputenc}
\usepackage{PTSansNarrow} 
\renewcommand*\familydefault{\sfdefault} %% Only if the base font of the document is to be sans serif
\usepackage[T1]{fontenc}
% Code highlighting
\usepackage{minted}
% Links
\usepackage[hidelinks]{hyperref}
% Courier
\usepackage{courier}

% Background color definition
\definecolor{bg}{rgb}{0.95,0.95,0.95}

% Code line numbers configuration
\renewcommand{\theFancyVerbLine}{\sffamily
  \textcolor[rgb]{0.5,0.5,1.0}{\scriptsize
     \oldstylenums{\arabic{FancyVerbLine}}}}


\makeatletter
\newcommand{\minted@write@detok}[1]{%
  \immediate\write\FV@OutFile{\detokenize{#1}}}%

\newcommand{\minted@FVB@VerbatimOut}[1]{%
  \@bsphack
  \begingroup
    \FV@UseKeyValues
    \FV@DefineWhiteSpace
    \def\FV@Space{\space}%
    \FV@DefineTabOut
    %\def\FV@ProcessLine{\immediate\write\FV@OutFile}% %Old, non-Unicode version
    \let\FV@ProcessLine\minted@write@detok %Patch for Unicode
    \immediate\openout\FV@OutFile #1\relax
    \let\FV@FontScanPrep\relax
%% DG/SR modification begin - May. 18, 1998 (to avoid problems with ligatures)
    \let\@noligs\relax
%% DG/SR modification end
    \FV@Scan}
    \let\FVB@VerbatimOut\minted@FVB@VerbatimOut

\renewcommand\minted@savecode[1]{
  \immediate\openout\minted@code\jobname.pyg
  \immediate\write\minted@code{\expandafter\detokenize\expandafter{#1}}%
  \immediate\closeout\minted@code}
\makeatother


\title{- III -\linebreak Estructuras de control de flujo}
\author{Ingeniería en Desarrollo de Contenidos Digitales\\ \textbf{Introducción a la Programación}\\ Prof. Álvaro Castro-Castilla}
\date{}

\begin{document}
\maketitle

\begin{center}
\includegraphics[scale=0.3,resolution=300]{images/utad.png}
\end{center}


\section{Esquemas básicos de las estructuras de control de flujo}
  \begin{enumerate}
  \item if-else

    \begin{minted}[linenos=true,bgcolor=bg]{c}
if (expression)
  statement
else if (expression)
  statement
else if (expression)
  statement
else if (expression)
  statement
else
  statement
    \end{minted}

  \item switch

    \begin{minted}[linenos=true,bgcolor=bg]{c}
switch (expression)
{
  case const-expr:
    statements
  case const-expr:
    statements
  case const-expr:
    statements
  case const-expr:
    statements
  default:
    statements
}
    \end{minted}
 
\pagebreak

  \item break

    \begin{minted}[linenos=true,bgcolor=bg]{c}
switch (expression)
{
  case const-expr:
    statements
    break; // Impide la ejecución de la siguiente línea
  case const-expr:
    statements
  defaut:
    break; // Por defecto no ejecuta nada
}
    \end{minted}

  \item while

    \begin{minted}[linenos=true,bgcolor=bg]{c}
while (expression)
  statement
    \end{minted}

  \item for

    \begin{minted}[linenos=true,bgcolor=bg]{c}
for (expr1; expr2; expr3)
  statement
    \end{minted}

  \item do-while

    \begin{minted}[linenos=true,bgcolor=bg]{c}
do
  statement
while (expression);
    \end{minted}

\section{continue y goto}
  \begin{enumerate}
  \item continue

    \begin{minted}[linenos=true,bgcolor=bg]{c}
for (i = 0; i < n; i++)
       if (a[i] < 0) /* saltarse los elementos negativos */
           continue;
       ... /* operar con los positivos */
    \end{minted}
    
  \item goto

    \begin{minted}[linenos=true,bgcolor=bg]{c}
for ( ... )
  for ( ... ) {
    ...
    if (error-condition)
      goto error;
} ...
error:
  /* código de error */
    \end{minted}

  \item Ejercicios
    \begin{enumerate}
      \item Escribe una función \textit{escape(destination,source)} que copie la cadena de caracteres \textit{source} a la \textit{destination}, modificando los caracteres no visibles por las secuencias de escape correspondientes. Es decir, el salto de línea por los caracteres \textbackslash{}n, el tabulador por \textbackslash{}t, etc. Debes usar un switch para esto.
      \item Escribe un programa que resuelva el laberinto que diseñaste en el tema 2 (ejercicio 9.1). Si el laberinto no es resoluble, diseña uno que lo sea. La solución deberá presentarse como una secuencia de movimientos (1: izquierda, 2: arriba, 3: derecha, 4: abajo) en forma de cadena de caracteres por la salida estándar (secuencia de números). 
      \item Escribe un programa que interprete una secuencia-solución de laberintos mostrando la secuencia de movimientos a la que corresponde (arriba, arriba, abajo, derecha...).
      \item Escribe el mismo programa agrupando los movimientos en la misma dirección (3 arriba, 1 derecha, 5 arriba...).
      \item \textbf{Tareas Especiales}.


        \texttt{Da Dubstep Mission
\newline
==================
\newline
          Dj Dijkstra está a la última con el Dubstep. Está que lo parte, y saca álbums cada semana. ¿Cómo lo hace? Sencillo: tiene un generador de dubstep de 400 GdB que consume más que un Hummer a gasolina. Tu misión es replicarlo para destrozale el negocio con versiones de Melendi. El truco de Dj Dijkstra es el siguiente:\newline
        \begin{enumerate}
        \item Lee una lírica del reputado compositor en la entrada estándar. Seguro que Google te puede ayudar con esto.
        \item Introduce una secuencia WUB, YUUUP o ZZZB (un número arbitrario de veces) entre cada sílaba, sacando la misma por salida estándar. Tendrás que averiguar como generar números aleatorios. Google al rescate, una vez más.
        \item El detector de sílabas funciona agrupando una vocal con una o más consonantes. Esto es lo más complicado. Encuentra tu propia solución para darle el toque especial a tu remezclador automático.
    \end{enumerate}}
    \end{enumerate}
  \end{enumerate}


  \end{enumerate}

\end{document}
