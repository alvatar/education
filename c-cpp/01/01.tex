\documentclass[a4paper,oneside]{article}
% Graphics
\usepackage{graphicx}
\usepackage{wrapfig}
% Font selection
\usepackage[utf8]{inputenc}
\usepackage{PTSansNarrow} 
\renewcommand*\familydefault{\sfdefault} %% Only if the base font of the document is to be sans serif
\usepackage[T1]{fontenc}
% Code highlighting
\usepackage{minted}
% Links
\usepackage[hidelinks]{hyperref}

% Background color definition
\definecolor{bg}{rgb}{0.95,0.95,0.95}

% Code line numbers configuration
\renewcommand{\theFancyVerbLine}{\sffamily
  \textcolor[rgb]{0.5,0.5,1.0}{\scriptsize
     \oldstylenums{\arabic{FancyVerbLine}}}}


\makeatletter
\newcommand{\minted@write@detok}[1]{%
  \immediate\write\FV@OutFile{\detokenize{#1}}}%

\newcommand{\minted@FVB@VerbatimOut}[1]{%
  \@bsphack
  \begingroup
    \FV@UseKeyValues
    \FV@DefineWhiteSpace
    \def\FV@Space{\space}%
    \FV@DefineTabOut
    %\def\FV@ProcessLine{\immediate\write\FV@OutFile}% %Old, non-Unicode version
    \let\FV@ProcessLine\minted@write@detok %Patch for Unicode
    \immediate\openout\FV@OutFile #1\relax
    \let\FV@FontScanPrep\relax
%% DG/SR modification begin - May. 18, 1998 (to avoid problems with ligatures)
    \let\@noligs\relax
%% DG/SR modification end
    \FV@Scan}
    \let\FVB@VerbatimOut\minted@FVB@VerbatimOut

\renewcommand\minted@savecode[1]{
  \immediate\openout\minted@code\jobname.pyg
  \immediate\write\minted@code{\expandafter\detokenize\expandafter{#1}}%
  \immediate\closeout\minted@code}
\makeatother


\title{- I -\linebreak Introducción general}
\author{Ingeniería en Desarrollo de Contenidos Digitales\\ \textbf{Introducción a la Programación}\\ Prof. Álvaro Castro-Castilla}
\date{}

\begin{document}
\maketitle

\begin{center}
\includegraphics[scale=0.3,resolution=300]{images/utad.png}
\end{center}


\section{Compilación y ejecución de un programa simple}
  \begin{enumerate}
  \item Compilación

    \begin{minted}[linenos=true,bgcolor=bg]{bash}
cc programa.c
    \end{minted}
  \item Ejecución

    \begin{minted}[linenos=true,bgcolor=bg]{bash}
./a.out
    \end{minted}
  \item Errores vs. Warnings
  \item Ejercicios
    \begin{enumerate}
    \item Busca 5 programas cortos en el lenguaje C. Asegúrate de que son en C y de que no superan las 50 líneas.
    \item Compílalos y ejecútalos en MSYS, Linux u OSX.
    \end{enumerate}
  \end{enumerate}

\section{Printf (a.k.a. ¡Hola mundo!)}
  \begin{enumerate}
  \item Código

    \begin{minted}[linenos=true,bgcolor=bg]{c}
#include <stdio.h>

main()
{
  printf("Hola Mundo\n");
}
    \end{minted}
  \item Partes
    \begin{enumerate}
      \item \#include
      \item main
      \item printf
    \end{enumerate}
  \item Ejercicios
    \begin{enumerate}
      \item Mostrar por pantalla el siguiente texto

      \begin{minted}[linenos=true,bgcolor=bg]{console}

                  /\/\,\,\ , 
                 /        ` \'\, 
                /               '/|_ 
               /                   / 
              /                   / 
             /                   ; 
             ;-""-.  ____       , 
            /      )'    `.     ' 
           (    o |        )   ; 
            ),'"""\    o   ;  : 
            ;\___  `._____/ ,-: 
           ;                 @ ) 
          /                `;-' 
       ,. `-.______________,| 
  ,(`._||         \__\__\__)| 
 ,`.`-   \        '.        | 
  `._  ) :          )______,;\_ 
     \    \_   _,--/       ,   `. 
      \     `--\   :      /      `. 
       \        \  ;     |         \ 
        `-._____ ;|      |       _,' 
                \/'      `-.----' \ 
                 /          \      \ 

      \end{minted}
    \end{enumerate}
  \end{enumerate}


\section{scanf}
  \begin{enumerate}
  \item Código

    \begin{minted}[linenos=true,bgcolor=bg]{c}
#include <stdio.h>

main ()
{
  char str [80];
  int i;

  printf ("Introduce tu nombre: ");
  scanf ("%79s",str);  
  printf ("Introduce tu edad: ");
  scanf ("%d",&i);
  printf ("Sr./Sra. %s , %d years old.\n",str,i);
}
    \end{minted}
  \item Partes
    \begin{enumerate}
    \item scanf
    \item variables (caracteres y números enteros)
    \item cadenas de formato (para printf y scanf)
    \end{enumerate}

  \item Ejercicios
    \begin{enumerate}
    \item Busca la lista de formateadores (format placeholders) de cadenas de caracteres.
    \item Programa en el que introduzcas un número entero y te lo muestre en formato hexadecimal.
    \item Programa en el que introduzcas un caracter y te diga su codificación ASCII.
    \item Programa que te pregunte 8 entradas de texto y muestre por pantalla los 8 textos unidos, empleando printf únicamente una vez para esto.
    \item Programa que muestre el medallero de curling en los juegos olímpicos en formato tabulado.
    \end{enumerate}
  \end{enumerate}

\section{Variables y expresiones aritméticas}
[continuará...]


\end{document}
