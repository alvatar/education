\documentclass[a4paper,oneside]{article}
% Graphics
\usepackage{graphicx}
\usepackage{wrapfig}
% Font selection
\usepackage[utf8]{inputenc}
\usepackage{PTSansNarrow} 
\renewcommand*\familydefault{\sfdefault} %% Only if the base font of the document is to be sans serif
\usepackage[T1]{fontenc}
% Code highlighting
\usepackage{minted}
% Links
\usepackage[hidelinks]{hyperref}
% Courier
\usepackage{courier}

% Background color definition
\definecolor{bg}{rgb}{0.95,0.95,0.95}

% Code line numbers configuration
\renewcommand{\theFancyVerbLine}{\sffamily
  \textcolor[rgb]{0.5,0.5,1.0}{\scriptsize
     \oldstylenums{\arabic{FancyVerbLine}}}}


\makeatletter
\newcommand{\minted@write@detok}[1]{%
  \immediate\write\FV@OutFile{\detokenize{#1}}}%

\newcommand{\minted@FVB@VerbatimOut}[1]{%
  \@bsphack
  \begingroup
    \FV@UseKeyValues
    \FV@DefineWhiteSpace
    \def\FV@Space{\space}%
    \FV@DefineTabOut
    %\def\FV@ProcessLine{\immediate\write\FV@OutFile}% %Old, non-Unicode version
    \let\FV@ProcessLine\minted@write@detok %Patch for Unicode
    \immediate\openout\FV@OutFile #1\relax
    \let\FV@FontScanPrep\relax
%% DG/SR modification begin - May. 18, 1998 (to avoid problems with ligatures)
    \let\@noligs\relax
%% DG/SR modification end
    \FV@Scan}
    \let\FVB@VerbatimOut\minted@FVB@VerbatimOut

\renewcommand\minted@savecode[1]{
  \immediate\openout\minted@code\jobname.pyg
  \immediate\write\minted@code{\expandafter\detokenize\expandafter{#1}}%
  \immediate\closeout\minted@code}
\makeatother


\title{- IV -\linebreak Funciones y estructura de los programas}
\author{Ingeniería en Desarrollo de Contenidos Digitales\\ \textbf{Introducción a la Programación}\\ Prof. Álvaro Castro-Castilla}
\date{}

\begin{document}
\maketitle

\begin{center}
\includegraphics[scale=0.3,resolution=300]{images/utad.png}
\end{center}


\section{Estructura de una función}
  \begin{enumerate}
  \item Estructura general

    \begin{minted}[linenos=true,bgcolor=bg]{c}
return-type function-name(arguments...)
{
  statements
}
    \end{minted}

  \item Función mínima (que no hace nada ni devuelve nada)

    \begin{minted}[linenos=true,bgcolor=bg]{c}
nothing() {}
    \end{minted}
  \end{enumerate}
 
\section{Estructura de un programa}
  \begin{enumerate}
    \newpage
    \item Función principal

    \begin{minted}[linenos=true,bgcolor=bg]{c}
#include <stdio.h>
#include <stdlib.h>

#define MAXOP   100
#define NUMBER  '0'

int getop(char []);
void push(double);
double pop(void);

/* Calculadora de notación polaca */
main()
{
  int type;
  double op2;
  char s[MAXOP];
  while ((type = getop(s)) != EOF) {
    switch (type) {
    case NUMBER:
      push(atof(s));
      break;
    case '+':
      push(pop() + pop());
      break;
    case '*':
      push(pop() * pop());
      break;
    case '-':
      op2 = pop();
      push(pop() - op2);
      break;
    case '/':
      op2 = pop();
      if (op2 != 0.0)
        push(pop() / op2);
      else
        printf("error: zero divisor\n");
      break;
    case '\n':
      printf("\t%.8g\n", pop());
      break;
    default:
      printf("error: unknown command %s\n", s);
      break;
    }
  }
  return 0;
}
    \end{minted}
    
  \newpage 
  \item push/pop

    \begin{minted}[linenos=true,bgcolor=bg]{c}
#define MAXVAL 100
int sp = 0;
double val[MAXVAL];

void push(double f)
{
  if (sp < MAXVAL)
    val[sp++] = f;
  else
    printf("error: stack full, can't push %g\n", f);
}

double pop(void)
{
  if (sp > 0)
    return val[--sp];
  else {
    printf("error: stack empty\n");
    return 0.0;
  }
}
    \end{minted}

  \item getch

    \begin{minted}[linenos=true,bgcolor=bg]{c}
#include <ctype.h>

int getch(void);
void ungetch(int);

int getop(char s[])
{
  int i, c;
  while ((s[0] = c = getch()) == ' ' || c == '\t')
    ;
  s[1] = '\0';
  if (!isdigit(c) && c != '.')
    return c;
  i = 0;
  if (isdigit(c))
    while (isdigit(s[++i] = c = getch()))
      ;
  if (c == '.')
    while (isdigit(s[++i] = c = getch()))
      ;
  s[i] = '\0';
  if (c != EOF)
    ungetch(c);
  return NUMBER;
}
    \end{minted}

  \item getch, ungetch

    \begin{minted}[linenos=true,bgcolor=bg]{c}
#define BUFSIZE 100
char buf[BUFSIZE];
int bufp = 0;
int getch(void)
{
  return (bufp > 0) ? buf[--bufp] : getchar();
}

void ungetch(int c)
{
  if (bufp >= BUFSIZE)
    printf("ungetch: too many characters\n");
  else
    buf[bufp++] = c;
}
    \end{minted}
  \end{enumerate}

\section{Ámbito de las variables}
  \begin{enumerate}
    \item extern

    \begin{minted}[linenos=true,bgcolor=bg]{c}
extern int sp;
extern double val[];
void push(double f) { ... }
double pop(void) { ... }
    \end{minted}

    \begin{minted}[linenos=true,bgcolor=bg]{c}
int sp = 0;
double val[MAXVAL];
    \end{minted}

    \item static

    \begin{minted}[linenos=true,bgcolor=bg]{c}
static char buf[BUFSIZE];
static int bufp = 0;

int getch(void) { ... }
void ungetch(int c) { ... }
    \end{minted}

    \newpage
    \item shadowing y ámbito

    \begin{minted}[linenos=true,bgcolor=bg]{c}
int x;
int y;

f(double x)
{
  double y;
  int z;
}

int g(void)
{
  return z * y; /* error, z no está en el ámbito */
}
    \end{minted}
  \end{enumerate}

\section{Archivos de cabecera}
  \begin{enumerate}
    \item calc.h

    \begin{minted}[linenos=true,bgcolor=bg]{c}
#define NUMBER '0'

void push(double);
double pop(void);
int getop(char []);
int getch(void);
void ungetch(int);
    \end{minted}
  \end{enumerate}

\section{Ejercicios}
  \begin{enumerate}
    \item Extiende el programa \textit{calculator.c} con las operaciones \textbf{sin}, \textbf{cos}, \textbf{exp}, \textbf{pow}.

    \item Escribe un intérprete para un lenguaje de pila con la siguientes características:

        \texttt{INSTRUCCIONES Y SINTAXIS
\newline
-
        \begin{enumerate}
        \item ADD: suma - sintaxis NUM1, NUM2, ADD
        \item MUL: multiplcación - sintaxis NUM1, NUM2, MUL
        \item LST: Left Shiftb(<<) - sintaxis: NUM1, NUM2, LST
        \item RST: Right Shift (>>) - sintaxis: NUM1, NUM2, RST
        \item OR: OR lógico - sintaxis: NUM1, NUM2, OR (produce un 1 si true, 0 si false)
        \item AND: AND lógico - sintaxis: NUM1, NUM2, AND (ídem)
        \item BOR: OR binario (a nivel de bits) - sintaxis: NUM1, NUM2, BOR
        \item BND: AND binario (a nivel de bits) - sintaxis: NUM1, NUM2, BND
    \end{enumerate}
PILA
\newline
-
\newline
El lenguaje funciona EXACTAMENTE igual que la calculadora que hemos creado en clase. Se basa en una pila en la que se van acumulando los valores y sobre la que se realiza la operación en el orden en el que se encuentran.
\newline
\newline
EJEMPLO DE PROGRAMA
\newline
-
\newline
10, 30, ADD, 2, LST, 3, 4, ADD, ADD
\newline
Producirá el equivalente al siguiente cálculo: (((10 + 30) << 2) + 4) + 3)}
  \end{enumerate}


\end{document}
