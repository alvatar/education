\documentclass[a4paper,oneside]{article}
% Graphics
\usepackage{graphicx}
\usepackage{wrapfig}
% Font selection
\usepackage[utf8]{inputenc}
\usepackage{PTSansNarrow} 
\renewcommand*\familydefault{\sfdefault} %% Only if the base font of the document is to be sans serif
\usepackage[T1]{fontenc}
% Code highlighting
\usepackage{minted}
% Links
\usepackage[hidelinks]{hyperref}

% Background color definition
\definecolor{bg}{rgb}{0.95,0.95,0.95}

% Code line numbers configuration
\renewcommand{\theFancyVerbLine}{\sffamily
  \textcolor[rgb]{0.5,0.5,1.0}{\scriptsize
     \oldstylenums{\arabic{FancyVerbLine}}}}


\makeatletter
\newcommand{\minted@write@detok}[1]{%
  \immediate\write\FV@OutFile{\detokenize{#1}}}%

\newcommand{\minted@FVB@VerbatimOut}[1]{%
  \@bsphack
  \begingroup
    \FV@UseKeyValues
    \FV@DefineWhiteSpace
    \def\FV@Space{\space}%
    \FV@DefineTabOut
    %\def\FV@ProcessLine{\immediate\write\FV@OutFile}% %Old, non-Unicode version
    \let\FV@ProcessLine\minted@write@detok %Patch for Unicode
    \immediate\openout\FV@OutFile #1\relax
    \let\FV@FontScanPrep\relax
%% DG/SR modification begin - May. 18, 1998 (to avoid problems with ligatures)
    \let\@noligs\relax
%% DG/SR modification end
    \FV@Scan}
    \let\FVB@VerbatimOut\minted@FVB@VerbatimOut

\renewcommand\minted@savecode[1]{
  \immediate\openout\minted@code\jobname.pyg
  \immediate\write\minted@code{\expandafter\detokenize\expandafter{#1}}%
  \immediate\closeout\minted@code}
\makeatother


\title{- II -\linebreak Tipos, Operadores y Expresiones}
\author{Ingeniería en Desarrollo de Contenidos Digitales\\ \textbf{Introducción a la Programación}\\ Prof. Álvaro Castro-Castilla}
\date{}

\begin{document}
\maketitle

\begin{center}
\includegraphics[scale=0.3,resolution=300]{images/utad.png}
\end{center}


\section{Tipos de datos}
  \begin{enumerate}
  \item Tipos básicos
    \begin{enumerate}
    \item char
    \item int
    \item float
    \item double
    \item char
    \end{enumerate}

  \item Modificadores de tamaño

    \begin{minted}[linenos=true,bgcolor=bg]{c}
short int my_short_int;
short my_second_short_int; /* sintaxis simplificada */
long int my_long_int;
long my_second_long_int; /* sintaxis simplificada */

long double my_long_double;
    \end{minted}

  \item Modificadores de signo

    \begin{minted}[linenos=true,bgcolor=bg]{c}
signed int a; /* número entero, con signo */
unsigned int b; /* número entero, sin signo */
unsigned char c; /* character (byte), sin signo */
    \end{minted}

  \item Printf y los tipos con modificador

    \begin{minted}[linenos=true,bgcolor=bg]{c}
printf("%l, %ul\n", 12l, 34ul);
    \end{minted}

  \item Ejercicios
    \begin{enumerate}
    \item Muestra por pantalla los valores mínimos y máximos de cada uno de los tipos y sus combinaciones de modificadores.
    \end{enumerate}
  \end{enumerate}

\section{Constantes}
  \begin{enumerate}
  \item Valores constantes
  \item Expresiones constantes
  \item Constantes enumeradas

    \begin{minted}[linenos=true,bgcolor=bg]{c}
enum boolean { NO, YES };

enum escapes {
  BELL = '\a',
  BACKSPACE = '\b',
  TAB = '\t',
  NEWLINE = '\n',
  VTAB = '\v',
  RETURN = '\r'
};
    \end{minted}
  \item Variables constantes

    \begin{minted}[linenos=true,bgcolor=bg]{c}
const double pi = 3.141592653589;
const char error_msg[] = "Error: ";

int strlen(const char[]);
    \end{minted}
  \end{enumerate}

\section{Operadores aritméticos}
  \begin{enumerate}
  \item +
  \item -
  \item *
  \item /
  \item \%
  \item Operador y asignación: op= (+=, -=, *=, /=, \%=)
  \end{enumerate}

\section{Operadores lógicos y relacionales}
  \begin{enumerate}
  \item >
  \item >=
  \item <
  \item <=
  \item ==
  \item !=
  \item ||
  \item \&\&
  \item Precedencia de los operadores (buscar table de precedencias de operadores en C)
  \item Valor numérico de una expresión booleana.
  \item Valores numéricos como expresión booleana.
  \end{enumerate}

\section{Conversiones entre tipos}
  \begin{enumerate}
  \item Conversión automática: de datos de un tamaño menor a otro mayor (no se produce pérdida de información).
  \item Conversión que perderán datos. El compilador normalmente mostrará un \textit{warning}.
  \item char a int, cuidado con los signos.
  \item Reglas de conversión automática
    \begin{enumerate}
    \item Si un operando es \textbf{long double}, convertir el otro a \textbf{long double}
    \item en caso contrario, si un operando es \textbf{double}, convertir el otro a \textbf{double}
    \item en caso contrario, si un operando es \textbf{float}, convertir el otro a \textbf{float}
    \item en caso contrario, convertir \textbf{char} y \textbf{short} a \textbf{int}
    \item por último, si un operando es \textbf{long}, convertir el otro a \textbf{long}
    \item Nota: los \textbf{floats} no se convierten automáticamente a \textbf{double}
    \end{enumerate}
  \item Conversiones y comparaciones entre \textbf{unsigned} y \textbf{signed}: dependen del procesador.
  \item Conversiones forzadas: \textit{cast}.

    \begin{minted}[linenos=true,bgcolor=bg]{c}
sqrt((double) n)
    \end{minted}
  \item Comprobar todo con los \textit{warnings} del compilador.
  \item La mejor regla: hacer sólo lo que tenga sentido.
  \item Extra: funciones en <ctype.h>
  \item Ejercicios
    \begin{enumerate}
    \item Escribe un programa que lea de la entrada estándar y convierta a decimal todos los números que encuentra, extrayéndolos en la salida estándar. Si encuentra un número en octal o en hexadecimal, debe ser convertido. Del mismo modo, detectará que los números contienen decimales y los mostrará de forma correcta.
    \end{enumerate}
  \end{enumerate}

\section{Operadores incremento y decremento}
  \begin{enumerate}
  \item Preincremento: ++n
  \item Postincremento: n++

    \begin{minted}[linenos=true,bgcolor=bg]{c}
int n = 0, a;
a = n++; // a = 0
n = 0;
a = ++n; // a = 1
    \end{minted}
  \end{enumerate}

\section{Operadores sobre bits}
  \begin{enumerate}
  \item \&
  \item |
  \item \^{}
  \item <<
  \item >>
  \item \~{}
  \item Ejercicios
    \begin{enumerate}
      \item Escribe un programa que muestre un número entero de 4 bits como patrón de 0s y 1s. Muestra los 0s y 1s agrupados en bytes. El número se introducirá con \textit{scanf}. Se mostrará la codificación del número en un entero con signo y sin signo, mostrando ambas codificaciones del número.
    \item Usa este programa para comprobar el efecto de cada uno de los operadores sobre bits, mostrando el número original y el resultado, con su patrón de 0s y 1s.
    \item Escribe una función \textit{set\_bit(int x, int\_bit, int value);} que modifique el valor del bit posicionado en \textit{n\_bit} al valor \textit{value}.
    \item Escribe una función \textit{invert\_bit(int x, int\_bit);} que invierta el valor del bit posicionado en \textit{bit}.
    \item Escribe una función que cuente el número de bits 0 en un int.
    \item Escribe una función que muestre el resultado de la operación \textit{x \&= (x-1)} sobre 10 números (tanto en decimal como en binario). ¿Qué ocurre con todos los números al aplicar esta operación?
    \end{enumerate}
  \end{enumerate}

\section{Expresiones condicionales}
  \begin{enumerate}
  \item El operador ternario condicional: ?:

    \begin{minted}[linenos=true,bgcolor=bg]{c}
if (a > b) {
    result = x;
} else {
    result = y;
}

/* Equivalente a: */
result = a > b ? x : y;
    \end{minted}
  \item Equivale a una expresión, con lo que las reglas de conversión se aplican como bloque.
  \end{enumerate}

\newpage
\section{Precedencia de los operadores}
  \begin{enumerate}
  \item Tabla ordenada de más precedencia a menos:

  \begin{tabular}{ l r }
() [] -> . & de izquierda a derecha \\
! ~ == -- = - *(derreferencia) sizeof & de \textbf{derecha a izquierda} \\
* / \% & de izquierda a derecha \\
+ - & de izquierda a derecha \\
<< >> & de izquierda a derecha \\
< <= > >= & de izquierda a derecha \\
== != & de izquierda a derecha \\
\& & de izquierda a derecha \\
\^{} & de izquierda a derecha \\
| & de izquierda a derecha \\
\&\& & de izquierda a derecha \\
|| & de izquierda a derecha \\
?: & de \textbf{izquierda a derecha} \\
= += -= *= /= \%= \&= \^{}= |= <<= >>= & de \textbf{derecha a izquierda} \\
, & de izquierda a derecha \\
\end{tabular}
  \item Ante la ambigüedad, emplear paréntesis.
  \item Ejercicios
    \begin{enumerate}
      \item Diseña manualmente con un editor de texto un laberinto de 32x32 caracteres. Los caracteres solo pueden ser 0 (vacío) y 1 (lleno). Puedes emplear dos caracteres distintos de forma que se vea mejor, pero uno debe significar lleno y otro vacío.
      \item Escribe un programa que codifique en un número entero cada una de las 32 filas de 0s y 1s, considerando estos como los bits que componen el número.
      \item Escribe un programa que descodifique los números enteros de 32 bits e imprima por pantalla los 0s y 1s que lo componen, reconstruyendo el laberinto.
    \end{enumerate}
  \end{enumerate}

\end{document}
